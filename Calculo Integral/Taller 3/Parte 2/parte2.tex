% !TEX root = ../main.tex

\documentclass[../main.tex]{subfiles}
\begin{document}
Cada una de las siguientes series es una serie telescopica o geometrica, o alguna serie relacionada cuya suma parcial puede ser simplificada. Demostrar que la serie converge al limite indicado.

\question $\sum\limits_{n = 1}^\infty \frac{2^n+3^n}{6^n} = \frac{3}{2}$

\begin{align*}
    \sum_{n = 1}^\infty \frac{2^n+3^n}{6^n} &= \sum_{n = 1}^\infty \frac{2^n}{6^n} + \sum_{n = 1}^\infty \frac{3^n}{6^n}\\
    &= \sum_{n = 1}^\infty \left(\frac{1}{3}\right)^n + \sum_{n = 1}^\infty \left(\frac{1}{2}\right)^n\\
    &=\frac{1}{3} \sum_{n = 0}^\infty \left(\frac{1}{3}\right)^n + \frac{1}{2}\sum_{n = 0}^\infty \left(\frac{1}{2}\right)^n\\
    &=\frac{1}{3} \frac{1}{1-\frac{1}{3}} + \frac{1}{2}\frac{1}{1-\frac{1}{2}}\\
    &= \frac{1}{3} \frac{1}{\frac{2}{3}} + \frac{1}{2} \frac{1}{\frac{1}{2}}\\
    &= \frac{1}{3} \cdot \frac{3}{2} + \frac{1}{2} \cdot 2\\
    &= \frac{1}{2} + 1\\
    &= \frac{3}{2}
\end{align*}


\question $\sum\limits_{n = 1}^\infty \frac{\sqrt{n+1} - \sqrt{n}}{\sqrt{n^2+n}} = 1$

Para empezar, vamos a descomponer la expresión $\frac{\sqrt{n+1} - \sqrt{n}}{\sqrt{n^2+n}}$ usando fracciones parciales.
\begin{align*}
    \frac{\sqrt{n+1} - \sqrt{n}}{\sqrt{n} \sqrt{n+1}} &= \frac{A}{\sqrt{n}} + \frac{B}{\sqrt{n+1}}\\
    \sqrt{n+1} - \sqrt{n} &= A\sqrt{n+1} + B\sqrt{n} 
\end{align*}
De esto, podemos deducir que $A = 1$ y $B = -1$. Por tanto, tendremos que:
\begin{align*}
    \frac{\sqrt{n+1} - \sqrt{n}}{\sqrt{n} \sqrt{n+1}} &= \frac{1}{\sqrt{n}} - \frac{1}{\sqrt{n+1}}\\
\end{align*}
Por lo que tendremos la siguiente suma telescopica:
\begin{align*}
    \sum\limits_{n = 1}^\infty \frac{\sqrt{n+1} - \sqrt{n}}{\sqrt{n^2+n}} &= \sum\limits_{n = 1}^\infty \frac{1}{\sqrt{n}} - \frac{1}{\sqrt{n+1}}\\
    &= \frac{1}{\sqrt{1}} - \lim_{n \to \infty} \frac{1}{\sqrt{n}}\\
    &= 1 - 0\\
    &= 1
\end{align*}

\question $\sum\limits_{n = 1}^\infty \frac{(-1)^{n-1}(2n+1)}{n(n+1)} = 1$\\

Empecemos por reducir la expresión con la ayuda de sumas parciales. Tendremos entonces la siguiente derivación:
\begin{align*}
    \frac{2n+1}{n(n+1)} &= \frac{A}{n} + \frac{B}{n+1}\\
    2n+1 &= A(n+1)+B(n)
\end{align*}
De donde saldrá que $A = B = 1$ y por tanto:
\begin{align*}
    \frac{2n+1}{n(n+1)} &= \frac{1}{n} + \frac{1}{n+1}\\
\end{align*}
Luego, combinando con $(-1)^{n-1}$ tendremos:
\begin{align*}
    (-1)^{n-1}\left(\frac{1}{n} + \frac{1}{n+1}\right) &= \frac{(-1)^{n-1}}{n} + \frac{(-1)^{n-1}}{n+1}\\
    &= \frac{(-1)^{n-1}}{n} - \frac{(-1)^n}{n+1}
\end{align*}
Lo que nos muestra que podemos usar las propiedades de una serie telescopica:
\begin{align*}
    \sum\limits_{n = 1}^\infty \frac{(-1)^{n-1}(2n+1)}{n(n+1)} &= \sum\limits_{n = 1}^\infty \frac{(-1)^{n-1}}{n} - \frac{(-1)^n}{n+1}\\
    &= \frac{(-1)^{1-1}}{1} - \lim_{n \to \infty} \frac{(-1)^{n-1}}{n}\\
    &= 1 - 0\\
    &= 1
\end{align*}


\question $\sum\limits_{n = 2}^\infty \frac{\log\left[\left(1 + \frac{1}{n}\right)(1+n)\right]}{\log(n^n) \log(n+1)^{n+1}} = \log_{2}(\sqrt{e})$

Desarrollando como sigue la serie original:
\begin{align*}
    \sum_{n = 2}^\infty \frac{\log\left[\left(1 + \frac{1}{n}\right)(1+n)\right]}{\log(n^n) \log(n+1)^{n+1}} &= \sum_{n =2}^\infty \frac{n \log\left(1+\frac{1}{n}\right) + \log(n+1)}{n \log(n) (n+1) \log(n+1)}\\
    &= \sum_{n = 2}^\infty \frac{n \log\left(\frac{n+1}{n}\right) + \log(n+1)}{n \log(n) (n+1) \log(n+1)}\\
    &= \sum_{n = 2}^\infty \frac{(n+1)\log(n+1) - n\log(n)}{n \log(n) (n+1) \log(n+1)}\\
    &= \sum_{n = 2}^\infty \frac{1}{n \log(n)} - \frac{1}{(n+1)\log(n+1)}
\end{align*}
Y dado que es una suma telescopica tendremos entonces:
\begin{align*}
    \sum_{n = 2}^\infty \frac{1}{n \log(n)} - \frac{1}{(n+1)\log(n+1)} &= \frac{1}{2 \log(2)} - \lim_{n \to \infty} \frac{1}{n \log(n)}\\
    &= \frac{1}{2 \log(2)} - 0\\
    &= \frac{\log(e)}{2 \log(2)}\\
    &= \frac{\log(\sqrt{e})}{\log(2)}\\
    &= \log_{2}(\sqrt{e})
\end{align*}

Usando la serie geometrica, y modificando con operaciones en ella, desarrollar las siguientes formulas:

\question $\sum\limits_{n = 1}^\infty n^3x^n = \frac{x^3+4x^2+x}{(1-x)^4}$\\

Para ello, tomaremos la serie geometrica original y derivaremos y multipliquemos por $x$:
\begin{align*}
    \sum_{n = 1}^\infty x^n &= \frac{1}{1-x}\\
    \sum_{n = 1}^\infty nx^{n-1} &= \frac{1}{(1-x)^2}\\
    \sum_{n = 1}^\infty nx^n &= \frac{x}{(1-x)^2}
\end{align*}
Volviendo a derivar y multiplicando por $x$
\begin{align*}
    \sum_{n = 1}^\infty n^2x^{n-1} &= \frac{(1-x)^2 + 2x(1-x)}{(1-x)^4}\\
    \sum_{n = 1}^\infty n^2x^{n-1} &= \frac{1+x}{(1-x)^3}\\
    \sum_{n = 1}^\infty n^2x^n &= \frac{x+x^2}{(1-x)^3}\\
\end{align*}
Y volviendo a repetir el proceso una ultima vez tendreos:
\begin{align*}
    \sum_{n = 1}^\infty n^3x^{n-1} &= \frac{(1-x)^3 (1+2x) + 3(x+x^2)(1-x)^2}{(1-x)^6}\\
    \sum_{n = 1}^\infty n^3x^{n-1} &= \frac{(1-x)(1+2x) + 3(x+x^2)}{(1-x)^4}\\
    \sum_{n = 1}^\infty n^3x^{n-1} &= \frac{1-x+2x-2x^2+3x^2+3x}{(1-x)^4}\\
    \sum_{n = 1}^\infty n^3x^{n-1} &= \frac{x^2+4x+1}{(1-x)^4}\\
    \sum_{n = 1}^\infty n^3x^{n} &= \frac{x^3+4x^2+x}{(1-x)^4}\\
\end{align*}

\question $\sum\limits_{n = 1}^\infty \frac{x^n}{n} = \log \frac{1}{1-x}$\\

Partiendo desde la serie geometrica e integrando tendremos:
\begin{align*}
    \int \sum\limits_{n = 0}^\infty x^n \, dx &= \int \frac{1}{1-x} \, dx\\
    \sum_{n = 0}^\infty \int x^n \, dx &= \int \frac{1}{1-x} \, dx\\
    \sum_{n = 0}^\infty \frac{x^{n+1}}{n+1} &= - \log(1-x) + C\\
    \sum_{n = 0}^\infty \frac{x^{n+1}}{n+1} &= \log\left(\frac{1}{1-x}\right) + C\\
\end{align*}
Si se reemplaza $x = 0$ podremos determinar fácilmente que $C = 0$, por lo que:
\begin{align*}
    \sum_{n = 1}^\infty \frac{x^n}{n} &= \log \frac{1}{1-x}
\end{align*}

\question $\sum_{n = 0}^\infty \frac{(n+1)(n+2)(n+3)}{3!} x^n = \frac{1}{(1-x)^4}$

Empezaremos desde la serie geometrica con un cambio de variable y derivaremos 3 veces:
\begin{align*}
    \sum_{n = -3}^\infty x^{n+3} &= \frac{1}{1-x}\\
    \sum_{n = -2}^\infty (n+3)x^{n+2} &= \frac{1}{(1-x)^2}\\
    \sum_{n = -1}^\infty (n+3)(n+2)x^{n+1} &= \frac{2(1-x)}{(1-x)^4}\\
    \sum_{n = -1}^\infty (n+3)(n+2)x^{n+1} &= \frac{2}{(1-x)^3}\\
    \sum_{n = 0}^\infty (n+3)(n+2)(n+1)x^{n} &= 2\frac{3(1-x)^2}{(1-x)^6}\\
    \sum_{n = 0}^\infty (n+3)(n+2)(n+1)x^{n} &= 3!\frac{1}{(1-x)^4}\\
    \sum_{n = 0}^\infty \frac{(n+3)(n+2)(n+1)}{3!}x^{n} &= \frac{1}{(1-x)^4}
\end{align*}

\question Dado que $\sum\limits_{n = 0}^\infty \frac{x^n}{n!} = e^x$ para todo $x$, encontrar la suma de la siguiente serie, asumiendo que está permitido manipular series infinitas como si fueran sumas finitas.
\begin{align*}
    \sum_{n = 2}^\infty \frac{n-1}{n!}
\end{align*}

Operaremos como prosigue:
\begin{align*}
    \sum_{n = 2}^\infty \frac{n-1}{n!} &= \sum_{n = 2}^\infty \frac{1}{(n-1)!} - \frac{1}{n!}\\
    &= \sum_{n =2}^\infty \frac{1}{(n-1)!} - \sum_{n = 2}^\infty \frac{1}{n!}\\
    &= \sum_{k = 1}^\infty \frac{1}{k!} - \sum_{n = 2}^\infty \frac{1}{n!}\\
    &= \sum_{k = 1}^\infty \frac{1}{k!} - \sum_{n = 1}^\infty \frac{1}{n!} + 1\\
    &= 1
\end{align*}

\question Dado que $\sum\limits_{n = 0}^\infty \frac{x^n}{n!} = e^x$ demostrar:
\begin{align*}
    \sum_{n = 1}^\infty \frac{n^2x^n}{n!} &= (x^2 + x)e^x
\end{align*}
Para esto, vamos a derivar y multiplicar por $x$ dos veces en la siguiente expresión:
\begin{align*}
    \sum_{n = 0}^\infty \frac{x^n}{n!} &= e^x\\
    \sum_{n = 0}^\infty \frac{nx^{n-1}}{n!} &= e^x\\
    \sum_{n = 0}^\infty \frac{nx^n}{n!} &= xe^x\\
    \sum_{n = 0}^\infty \frac{n^2x^{n-1}}{n!} &= e^x(x+1)\\
    \sum_{n = 0}^\infty \frac{n^2x^n}{n!} &= e^x(x^2+x)\\
    \sum_{n = 1}^\infty \frac{n^2x^n}{n!} &= e^x(x^2+x)\\
\end{align*}
\end{document}