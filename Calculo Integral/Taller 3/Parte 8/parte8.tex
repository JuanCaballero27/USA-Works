% !TEX root = ../main.tex

\documentclass[../main.tex]{subfiles}
\begin{document}
En los siguientes ejercicios, se usa una serie de potencias para definir una función. Determinar los intervalos de convergencia y demostrar que satisface la ecuación diferencial indicada. 

\question $f(x) = \sum\limits_{n = 0}^\infty \frac{x^n}{(n!)^2}$ y $xy'' + y' - y = 0$.\\

Si se aplica el test de la razón tendremos:
\begin{align*}
    \lim_{n \to \infty} \left|\frac{a_{n+1}}{a_n}\right| &= \lim_{n \to \infty} \left|\frac{\frac{x^{n+1}}{((n+1)!)^2}}{\frac{x^n}{(n!)^2}}\right|\\
    &= \lim_{n \to \infty} \left|\frac{x^{n+1} (n!)^2}{x^n ((n+1)!)^2}\right|\\
    &= \lim_{n \to \infty} \frac{|x|}{(n+1)^2}\\
    &= 0
\end{align*}
Por lo que la serie converge para todo $x \in \mathbb{R}$. Luego, si derivamos la función:
\begin{align*}
    \frac{d}{dx} f(x) &= \sum_{n = 1}^\infty \frac{nx^{n-1}}{(n!)^2}\\
    \frac{d^2}{dx^2} f(x) &= \sum_{n = 2}^\infty \frac{n(n-1)x^{n-2}}{(n!)^2}\\
\end{align*}

Por lo que si evaluamos la expresión del lado izquierdo:
\begin{align*}
    xy'' + y'-y &= x\sum_{n = 2}^\infty \frac{n(n-1)x^{n-2}}{(n!)^2} + \sum_{n = 1}^\infty \frac{nx^{n-1}}{(n!)^2} - \sum_{n = 0}^\infty \frac{x^n}{(n!)^2}\\
    &= \sum_{n = 2}^\infty \frac{n(n-1)x^{n-1}}{(n!)^2} + \sum_{n = 1}^\infty \frac{nx^{n-1}}{(n!)^2} - \sum_{n = 0}^\infty \frac{x^n}{(n!)^2}\\
    &= \sum_{n = 2}^\infty \frac{n(n-1)x^{n-1}}{(n!)^2} + \sum_{n = 2}^\infty \frac{nx^{n-1}}{(n!)^2} + 1 - 1 - \sum_{n = 1}^\infty \frac{x^n}{(n!)^2}\\
    &= \sum_{n = 2}^\infty \frac{nx^{n-1}(n-1+1)}{(n!)^2} - \sum_{n = 1}^\infty \frac{x^n}{(n!)^2}\\
    &= \sum_{n = 2}^\infty \frac{n^2x^{n-1}}{(n!)^2} - \sum_{n = 1}^\infty \frac{x^n}{(n!)^2}\\
    &= \sum_{n = 1}^\infty \frac{(n+1)^2x^{n}}{((n+1)!)^2} - \sum_{n = 1}^\infty \frac{x^n}{(n!)^2}\\
    &= \sum_{n = 1}^\infty \frac{x^{n}}{(n!)^2} - \sum_{n = 1}^\infty \frac{x^n}{(n!)^2}\\
    &= 0
\end{align*}

\question $f(x) = \sum\limits_{n = 0}^\infty \frac{(-1)^n 2^{2n} x^{2n}}{(2n)!}$ y $y''+4y=0$.\\

Para empezar, usaremos el criterio de la razón:
\begin{align*}
    \lim_{n \to \infty} \left|\frac{a_{n+1}}{a_n}\right| &= \lim_{n \to \infty} \left|\frac{\frac{(-1)^{n+1} 2^{2n+2} x^{2n+2}}{(2n+2)!}}{\frac{(-1)^n 2^{2n} x^{2n}}{(2n)!}}\right|\\
    &= \lim_{n \to \infty} \left|\frac{(-1)^{n+1} 2^{2n+2} x^{2n+2} (2n)!}{(-1)^n 2^{2n} x^{2n} (2n+2)!}\right|\\
    &= \lim_{n \to \infty} \frac{4x^2}{(2n+2)(2n+1)}\\
    &= 0
\end{align*}
Por lo que la serie converge para todo $x \in \mathbb{R}$. Derivando la función:
\begin{align*}
    \frac{d^2}{dx^2} f(x) &= \sum_{n = 2}^\infty \frac{(-1)^n 2^{2n} (2n) (2n-1) x^{2n-2}}{(2n)!}
\end{align*} 
Y manipulando la ecuación:
\begin{align*}
    y'' + 4y &=  \sum_{n = 1}^\infty \frac{(-1)^n 2^{2n} (2n) (2n-1) x^{2n-2}}{(2n)!} + 4\sum_{n = 0}^\infty \frac{(-1)^n 2^{2n} x^{2n}}{(2n)!}\\
    &= \sum_{n = 1}^\infty \frac{(-1)^n 2^{2n} x^{2n-2}}{(2n-2)!} + 4\sum_{n = 0}^\infty \frac{(-1)^n 2^{2n} x^{2n}}{(2n)!}\\
    &= \sum_{n = 1}^\infty \frac{(-1)^n 2^{2n} x^{2n-2}}{(2n-2)!} + \sum_{n = 0}^\infty \frac{(-1)^n 2^{2n+2} x^{2n}}{(2n)!}\\
    &= \sum_{n = 0}^\infty \frac{(-1)^{n+1} 2^{2n+2} x^{2n}}{(2n)!} + \sum_{n = 0}^\infty \frac{(-1)^n 2^{2n+2} x^{2n}}{(2n)!}\\
    &= \sum_{n = 0}^\infty \frac{(-1)^{n+1} 2^{2n+2} x^{2n}}{(2n)!} + \sum_{n = 0}^\infty \frac{(-1)^n 2^{2n+2} x^{2n}}{(2n)!}\\
    &= \sum_{n = 0}^\infty (-1)^n \left[\frac{-(2^{2n+2} x^{2n}) + (2^{2n+2}x^{2n})}{(2n)!}\right]\\
    &= \sum_{n = 0}^\infty (-1)^n 0\\
    &= 0
\end{align*}

\question $f(x) = x + \sum\limits_{n = 0}^\infty \frac{(3x)^{2n+1}}{(2n+1)!}$ y $y'' = 9(y-x)$.\\

Si aplicamos el test de la razón:
\begin{align*}
    \lim_{n \to \infty} \left|\frac{a_{n+1}}{a_n}\right| &= \lim_{n \to \infty} \left|\frac{\frac{(3x)^{2n+3}}{(2n+3)!}}{\frac{(3x)^{2n+1}}{(2n+1)!}}\right|\\
    &= \lim_{n \to \infty} \left|\frac{(3x)^{2n+3} (2n+1)!}{(3x)^{2n+1} (2n+3)!}\right|\\
    &= \lim_{n \to \infty} \frac{(3x)^2}{(2n+3)(2n+2)}\\
    &= 0
\end{align*}
Por lo que la serie será convergente para todo $x \in \mathbb{R}$. Luego, si derivamos la función:
\begin{align*}
    \frac{d^2}{dx^2} &= \sum_{n = 1}^\infty \frac{3^{2n+1} \cdot (2n+1)(2n) x^{2n-1}}{(2n+1)!}\\
    &= 3 \cdot \sum_{n = 1}^\infty \frac{3^{2n} x^{2n-1}}{(2n-1)!}\\
    &= 3 \cdot \sum_{n = 0}^\infty \frac{3^{2n+2} x^{2n+1}}{(2n+1)!}\\
    &= 9 \cdot \sum_{n = 0}^\infty \frac{3^{2n+1} x^{2n+1}}{(2n+1)!}
\end{align*}
Luego, si partimos del lado derecho de la igualdad:
\begin{align*}
    9(y-x) &= 9\left[x + \sum_{n = 0}^\infty \frac{(3x)^{2n+1}}{(2n+1)!} - x\right]\\
    &= 9 \sum_{n = 0}^\infty \frac{(3x)^{2n+1}}{(2n+1)!}\\
    &= 3^2 \sum_{n = 0}^\infty \frac{(3x)^{2n+1}}{(2n+1)!}\\
    &= \frac{d^2}{dx^2} f(x)
\end{align*}


Las funciones:
\begin{align*}
    J_0(x) &= \sum_{n = 0}^\infty (-1)^n \frac{x^{2n}}{(n!)^2 2^{2n}}, & J_1(x) &= \sum_{n = 0}^\infty (-1)^n \frac{x^{2n+1}}{n! (n+1)! 2^{2n+1}}
\end{align*}
Son llamadas \textit{funciones de Bessel de primer tipo} de orden $0$ y $1$ respectivamente. Estas funciones suelen aparecer en varios problemas en matemáticas puras y aplicadas. Demostrar:

\question Ambas series convergen para todo $x\in \mathbb{R}$
\begin{proof}
    Para $J_0$ aplicaremos el criterio de la razón:
    \begin{align*}
        \lim_{n \to \infty} \left|\frac{a_{n+1}}{a_n}\right| &= \lim_{n \to \infty} \left|\frac{\frac{(-1)^{n+1} x^{2n+2}}{((n+1)!)^2 2^{2n+2}}}{\frac{(-1)^nx^{2n}}{(n!)^2 2^{2n}}}\right|\\
        &= \lim_{n \to \infty} \left|\frac{x^{2n+2} (n!)^2 2^{2n}}{x^{2n} ((n+1)!)^2 2^{2n+2}}\right|\\
        &= \lim_{n \to \infty} \left|\frac{x^2}{4(n+1)^2 }\right|\\
        &= 0
    \end{align*}
    Y si hacemos lo mismo para $J_1$ entonces:
    \begin{align*}
        \lim_{n \to \infty} \left|\frac{a_{n+1}}{a_n}\right| &= \lim_{n \to \infty} \left|\frac{\frac{(-1)^{n+1}x^{2n+3}}{(n+1)! (n+2)! 2^{2n+3}}}{\frac{(-1)^nx^{2n+1}}{n! (n+1)! 2^{2n+1}}}\right|\\
        &= \lim_{n \to \infty} \left|\frac{x^{2n+3} n! (n+1)! 2^{2n+1}}{x^{2n+1} (n+1)! (n+2)! 2^{2n+3}}\right|\\
        &= \lim_{n \to \infty} \left|\frac{x^2}{4(n+1)(n+2)}\right|\\
        &= 0
    \end{align*}
    Por lo que ambas series convergen para todo $x \in \mathbb{R}$.
\end{proof}

\question Demostrar que $J'_0(x) = -J_1(x)$
\begin{proof}
    Si derivamos $J'0(x)$ tendremos:
    \begin{align*}
        \frac{d}{dx} J_0(x) &= \sum_{n = 1}^\infty (-1)^n \frac{2n x^{2n-1}}{(n!)^2 2^{2n}}\\
        &= \sum_{n = 1}^\infty (-1)^n \frac{x^{2n-1}}{(n-1)! n! 2^{2n-1}}\\
        &= \sum_{n = 0}^\infty (-1)^{n+1} \frac{x^{2n+1}}{n! (n+1)! 2^{2n+1}}\\
        &= (-1) \sum_{n = 0}^\infty (-1)^{n} \frac{x^{2n+1}}{n! (n+1)! 2^{2n+1}}\\
        &= - J_1(x)
    \end{align*}
\end{proof}
\end{document}
