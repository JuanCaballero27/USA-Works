\documentclass[12pt,a4paper,oneside]{memoir}
\usepackage{pstricks-add}
\usepackage[top=1cm,left=1cm,right=1.5cm,bottom=2cm]{geometry}
\usepackage[spanish]{babel}
\usepackage[utf8]{inputenc}
\usepackage[colorlinks=true,urlcolor=magenta,citecolor=red,linkcolor=violet,bookmarks=true]{hyperref}
\usepackage[sort&compress,round,comma,authoryear]{natbib}
\usepackage{makeidx}
\usepackage{lscape}
\usepackage{pdflscape}
\usepackage{epstopdf}
\usepackage{booktabs}
\usepackage{pdfpages}
\usepackage{textcomp}
\usepackage[many]{tcolorbox}
\usepackage{empheq}
\usepackage{tasks}
\usepackage{array}
\usepackage{tikz}
\usepackage[T1]{fontenc}
\usepackage{ae}
\usepackage{lipsum}
\usepackage{indentfirst}
\usepackage{graphicx}
\usepackage{subfig}
\usepackage{float}
\usepackage{blindtext}
\usepackage{tabularx}
\usepackage{ragged2e}
\usepackage{xcolor}
\usepackage{multirow}
\usepackage{bookmark}
\usepackage{pgfplots}
\usepackage{amsmath,amssymb,amsthm}
\usepackage{lastpage}
\usepackage{epigraph}
\usepackage{enumerate}
\usepackage{enumitem}
\usepackage{mathrsfs}
\usepackage{tikz}
\usepackage{pgfplots}
\pgfplotsset{compat=1.15}

\usetikzlibrary{arrows}
\usepackage{subfiles} % Insert the commands.tex file which contains the majority of the structure behind the template
\pgfplotsset{compat=1.15}

\newlist{questions}{enumerate}{3}
\setlist[questions]{label=\arabic*.}
\newcommand{\question}{\item}
\setlist[enumerate,1]{% (
leftmargin=*, itemsep=12pt, label={\textbf{\arabic*.)}}}

\newlist{partes}{enumerate}{3}
\setlist[partes]{label=(\alph*)}
\newcommand{\parte}{\item}
%---
\newlist{subpartes}{enumerate}{3}
\setlist[subpartes]{label=\roman*)}
\newcommand{\subparte}{\item}


\newcommand*\circled[1]{\tikz[baseline=(char.base)]{\node[shape=circle,draw,inner sep=2pt] (char) {#1};}}
\newcommand{\instituto}{Universidad Sergio Arboleda}
\newcommand{\curso}{Cálculo Integral}
\newcommand{\professor}{Fabio Molina}
\newcommand{\disciplina}{Matemáticas}
\newcommand{\titulo}{Taller 2}
\newcommand{\alumnoI}{Juan Sebastián Caballero Bernal}
\newcommand{\alumnoII}{Luz Ángela Orjuela Nieto}
\pagestyle{plain}
\linespread{1.5}
\pagestyle{empty}
\newtheorem*{definition*}{Definición}
\newtheorem*{theorem*}{Teorema}
\newtheorem*{axiom*}{Postulado}
\newtheorem{theorem}{Teorema}[section]
\renewcommand*{\proofname}{\textbf{Demostración}}

\begin{document}
%%%%%%%%%%%%%%%%%%%%%%%%%%%%%%%%%%%%%%%%%%%%%%%%%%%%%%%%
%                      Emcabezado                     %
%%%%%%%%%%%%%%%%%%%%%%%%%%%%%%%%%%%%%%%%%%%%%%%%%%%%%%%%
\begin{table}[H]
\centering
\begin{tabular*}{\textwidth}{l@{\extracolsep{\fill}}l@{\extracolsep{\fill}}}
    \begin{tabular}[l]{@{}l@{}}
        \textbf{\instituto}\\
        \textbf{Disciplina: \disciplina}\\
        \textbf{Profesor: \professor}\\ 
    \end{tabular} & 
    \begin{tabular}[l]{@{}l@{}}
        {\curso}\\
        {\alumnoI}\\
    \end{tabular}
\end{tabular*}
\end{table}
\begin{center}
\rule[2ex]{\textwidth}{1pt}
{\Large{\titulo}}
\end{center}
\rule[2ex]{\textwidth}{1pt}

\section*{Sección 10.4}
En los siguientes ejercicios, una secuencia $\{f(n)\}$ es definida por la formula dada. En cada caso, determinar si la secuencia converge o diverge, y dado el caso, determinar el limite la serie.
\begin{questions}[label=\protect\circled{\bfseries\arabic*}]

\subfile{Parte 1/parte1.tex}

\section*{Sección 10.9}
\subfile{Parte 2/parte2.tex}

\section*{Sección 10.14}
\subfile{Parte 3/parte3.tex}

\section*{Sección 10.16}
\subfile{Parte 4/parte4.tex}

\section*{Sección 10.20}
\subfile{Parte 5/parte5.tex}

\section*{Sección 10.24}
\subfile{Parte 6/parte6.tex}

\section*{Sección 11.13}
\subfile{Parte 7/parte7.tex}

\section*{Sección 11.16}
\subfile{Parte 8/parte8.tex}

\end{questions}
\end{document}