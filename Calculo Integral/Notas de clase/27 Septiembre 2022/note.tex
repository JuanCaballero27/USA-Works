\documentclass[12pt,a4paper,oneside]{memoir}
\usepackage{notesconfig}


% \pagestyle{plain}
\newcommand{\instituto}{Universidad Sergio Arboleda}
\newcommand{\curso}{Calculo Integral}
\newcommand{\professor}{Fabio Molina}
\newcommand{\disciplina}{Matemáticas}
\newcommand{\titulo}{27 Septiembre 2022}
\newcommand{\alumnoI}{Juan Sebastián Caballero Bernal}
\newcommand{\alumnoII}{Luz Ángela Orjuela Nieto}

\linespread{1.5}

\begin{document}
%-----------------------------------------------------------
%                       Encabezado
%-----------------------------------------------------------
\begin{table}[H]
\centering
\begin{tabular*}{\textwidth}{l@{\extracolsep{\fill}}l@{\extracolsep{\fill}}}
    \begin{tabular}[l]{@{}l@{}}
        \textbf{\instituto}\\
        \textbf{Disciplina: \disciplina}\\
        \textbf{Profesor: \professor}\\ 
    \end{tabular} & 
    \begin{tabular}[l]{@{}l@{}}
        {\curso}\\
        {\alumnoI}\\
        % {\alumnoII}\\
    \end{tabular}
\end{tabular*}
\end{table}
\begin{center}
\rule[2ex]{\textwidth}{1pt}

{\Large{\titulo}}
\end{center}
\rule[2ex]{\textwidth}{1pt}
%--------------------------------------------------------
%                       Contenido                      
%--------------------------------------------------------
\section*{Funciones iguales}
Anteriormente hemos visto un teorema muy importante y como dicho teorema nos ayuda a generar resultados como un criterio para minimos y máximos. La idea de funciones iguales hasta orden $n$ busca generalizar este concepto más allá de los Polinomios de Taylor, pero a su vez nos dará una herramienta potente para trabajar de nuevo.

\begin{definition}[Funciones iguales de orden $n$]
    Sean $f$ y $g$ dos funciones de variable real, diremos que $f$ es igual a $g$ hasta orden $n$ en $a$ si y solo si:
    \begin{align*}
        \lim_{x \to a} \frac{f(x)-g(x)}{(x-a)^k} = 0
    \end{align*}
    Para $0 \le k \le n$.
\end{definition}
Esta definición es una generalización del teorema sobre limite de la diferencia del Polinomio de Taylor y su función. En general, esta definición resultar ser útil para cualquier tipo de funciones, puesto que más adelante nos permitirá generar más fácilmente una formula para polinomios de Taylor.

\section*{Igualdad entre Polinomios}
Con todo en orden, es hora de probar un teorema que de primera vista parece evidente pero que vendrá de ayuda más adelante:
\begin{theorem}
    Sean $P$ y $Q$ dos polinomios de variable real con grado menor o igual que $n$, de forma que $P$ es igual hasta orden $n$ a $Q$ en un punto $c$. Podemos concluir que $P = Q$.
\end{theorem}
\begin{proof}
    Para esto, tengamos en cuenta que si son iguales hasta orden $n$ podremos de forma recursiva con el orden del limite. Por ejemplo, si $P(x) = a_0 + a_1(x-c)+a_2(x-c)^2 + \dots + a_n(x-c)^n$ y $Q(x) = b_0 + b_1(x-c) + b_2(x-c)^2 + \dots + b_n(x-c)^n$ tendremos para cuando $k = 0$:
    \begin{align*}
        \lim_{x \to c} \frac{P(x) - Q(x)}{(x-c)^0} &= \lim_{x \to c} P(x) - Q(x)\\
        &=\lim_{x \to c} \left[(a_0 - b_0) + \sum_{i = 1}^n (a_i - b_i) (x-c)^i\right]\\
        &= a_0 - b_0 = 0
    \end{align*}
    Por lo que concluimos que $a_0 = b_0$. Luego, si $k = 1$ tendremos:
    \begin{align*}
        \lim_{x \to c} \frac{P(x) - Q(x)}{x-c} &= \lim_{x \to c} \frac{(a_0 - b_0) + (a_1 - b_1)(x-c) + \sum_{i = 2}^n (a_i - b_i)(x-c)^i}{x-c}\\
        &= \lim_{x \to c} \frac{(a_1 - b_1)(x-c)}{x-c} + \frac{1}{x-c} \cdot \sum_{i = 2}^n (a_i - b_i)(x-c)^i\\
        &= \lim_{x \to c} (a_1-b_1) + \sum_{i = 2}^n (a_i - b_i)(x-c)^{i-1}\\
        &= a_1 - b_1 = 0
    \end{align*}
    Por lo que ahora $a_1 = b_1$. En general si continuamos con este proceso iterativamente, para $k$ en general tendremos:
    \begin{align*}
        \lim_{x \to c}\frac{P(x)-Q(x)}{(x-c)^k} &= \lim_{x \to c} \frac{\sum_{i = 0}^{k-1} (a_i-b_i)(x-c)^i + (a_k + b_k)(x-c)^k + \sum_{i = k+1}^n (a_i - b_i)(x-c)^i}{(x-c)^k}\\
        &= \lim_{x \to c} \frac{(a_k-b_k)(x-c)^k}{(x-c)^k} + \frac{1}{(x-c)^k} \sum_{i = k+1}^n (a_i - b_i)(x-c)^i\\
        &= \lim_{x \to c} (a_k - b_k) + \sum_{i = k+1}^n (a_i - b_i)(x-c)^{i-k}\\
        &= a_k - b_k = 0
    \end{align*}
    Por lo que en general $a_k = b_k$ lo que nos permite concluir que $P = Q$.
\end{proof}

Este teorema, tan bello y \textit{relativamente} sencillo de probar da paso a un corolario que nos dará información sobre el Polinomio de Taylor:
\begin{corollary}
    Sea $f$ una función de forma que $P_{n, a, f}$ existe y además existe un polinomio $P$ de forma que $f$ es igual hasta orden $n$ en a $P$ en $a$, entones $P = P_{n, a, f}$.
\end{corollary}
\begin{proof}
    Dado que ya se ha demostrado que el polinomio de Taylor de orden $n$ para $f$ en $a$ es igual hasta orden $n$ a $f$ en $a$, tendremos:
    \begin{align*}
        \lim_{x \to a} \frac{f(x) - P_{n, a, f}(x)}{(x-a)^k} &= 0
    \end{align*} 
    Para cualquier $k$ entre $0$ y $n$. Luego, tendremos por hipotesís:
    \begin{align*}
        \lim_{x \to a} \frac{f(x) - P(x)}{(x-a)^k} &= 0
    \end{align*}
    Igualando ambas expresiones:
    \begin{align*}
        \lim_{x \to a} \frac{f(x) - P_{n, a, f}(x)}{(x-a)^k} &= \lim_{x \to a} \frac{f(x) - P(x)}{(x-a)^k}\\
        \lim_{x \to a} \frac{f(x) - P_{n, a, f}(x)}{(x-a)^k} - \lim_{x \to a} \frac{f(x) - P(x)}{(x-a)^k} &= 0\\
        \lim_{x \to a} \frac{f(x) - P_{n, a, f}(x)}{(x-a)^k} - \frac{f(x) - P(x)}{(x-a)^k} &= 0\\
        \lim_{x \to a} \frac{f(x) - P_{n, a, f}(x) - f(x) + P(x)}{(x-a)^k} &= 0\\
        \lim_{x \to a} \frac{P(x) - P_{n, a, f}(x)}{(x-a)^k} &= 0\\
    \end{align*}
    Por lo que $P$ es igual hasta orden $n$ a $P_{n, a, f}$ y por el teorema anterior tendremos que $P = P_{n, a, f}$.
\end{proof}

\begin{remark}
    La prueba anterior refleja un poco sobre la naturaleza de la relación \textit{Ser igual hasta orden $n$ en $a$ a..}. Con un poco más de esfuerzo se puede demostrar que la relación es una relación de equivalencia(Reflexiva, Simetrica y Transitiva).
\end{remark}

\begin{example}
    Recordemos que determinar el polinomio de Taylor para la función $f(x) = \arctan(x)$ era un proceso tedioso. Aquí miraremos una forma de determinar de manera más \textit{directa} dicho polinomio de Taylor para $a = 0$.

    Empecemos notando que $\arctan(x) = \int_{0}^x \frac{1}{1+t^2} dt$, y dentro de la expresión de adentro podremos realizar una división síntetica para terminar con algo así:
    \begin{align*}
        \arctan(x) &= \int_{0}^x 1 - t^2 + t^4 + \dots +(-1)^n t^{2n} + \frac{(-1)^{n+1}t^{2n+2}}{1+t^2} dt\\
        &= t - \frac{t^3}{3} + \frac{t^5}{5} + \dots + (-1)^n \frac{t^{2n+1}}{2n+1} + \int_{0}^x \frac{(-1)^{n+1}t^{2n+2}}{1+t^2} dt 
    \end{align*}
    Ahora, notese que en la última expresión tenemos un polinomio con una integral. Si la integral tiende a $0$ habremos encontrado el polinomio de Taylor de grado $2n+1$ para $\arctan(x)$. Por ello, analizar la expresión será importante.\\

    Notese que:
    \begin{align*}
        \int_{0}^x \frac{(-1)^{n+1}t^{2n+2}}{1+t^2} dt &\le \left|\int_{0}^x \frac{(-1)^{n+1}t^{2n+2}}{1+t^2} dt \right|\\
        &= \left|\int_{0}^x \frac{t^{2n+2}}{1+t^2} dt \right|\\
        &\le \left|\int_{0}^x \frac{t^{2n+2}}{1} dt  \right|\\
        &= \left|\int_{0}^x t^{2n+2} dt\right|\\
        &= \left|\frac{x^{2n+3}}{2n+3}\right|\\
        &= \frac{ |x|^{2n+3}}{2n+3}
    \end{align*}
    Luego, sin importar el valor de $x$ se puede tomar $n$ suficientemente grande para hacer que $\frac{ |x|^{2n+3}}{2n+3}$ tienda a $0$ con lo que $\int_{0}^x \frac{(-1)^{n+1}t^{2n+2}}{1+t^2} dt $ tenderá a $0$ y luego, es evidente que $\arctan(x)$ y el Polinomio escrito son iguales hasta orden $n$ en $0$. Por tanto:
    \begin{align*}
        P_{2n+1, 0, f}(x) &= \sum_{k = 0}^n (-1)^k\frac{x^{2k+1}}{2k+1}
    \end{align*}
    E incluso esto permite determinar los valores para las derivadas de $\arctan(x)$ evaluadas en $0$ con la siguiente formula(¿Por qué?):
    \begin{align*}
        \frac{d^{2k+1}}{dx^{2k+1}} \arctan(x) &= (-1)^k\frac{k!}{2k+1}
    \end{align*}
\end{example}

\section*{Inicios de la formula para el polinomio de Taylor}
Recordemos que nuestra meta es encontrar una forma sencilla de determinar un polinomio de Taylor. Pues, dentro de un ejercicio practico podremos hacer algo así.\\

Para $\int_{a}^x \frac{d}{dx} f(t) dt$ al evaluarse por el \textit{Teorema fundamental del cálculo} tendremos:
\begin{align*}
    \int_{a}^x \frac{d}{dx} f(t) dt &= f(x) - f(a)
\end{align*}
Luego, despejando $f(x)$ tendremos:
\begin{align*}
    f(x) &= f(a) + \int_{a}^x \frac{d}{dx} f(t) dt
\end{align*}
Notese en este punto que $f(a)$ es el Polinomio de grado $0$ en $a$ para $f$. Luego, si integramos por partes haciendo que $u = \frac{d}{dx} f(t)$ y $dv = dt$, para que tengamos que $du = \frac{d^2}{dx^2} f(t) dt$ y $v = t-x$(¿Por qué no aplicar $v = t$?), tendremos:
\begin{align*}
    f(x) &= f(a) + \int_{a}^x \frac{d}{dx} f(t) dt\\
    &= f(a) + \left(\frac{d}{dx} f(t) (t -x)|_{a}^{x}\right) + \int_{a}^x (x-t) \frac{d^2}{dx^2}f(t) dt\\
    &= f(a) + \frac{d}{dx} f(a)(x-a) + \int_{a}^x (x-t) \frac{d^2}{dx^2} f(t)dt
\end{align*}
Lo que viene siendo el polinomio de Taylor de grado $1$ junto con otra integral. Este proceso recursivo nos llevará más adelante a una formula bonita y elegante para determinar en general el polinomio de Taylor.
\end{document}