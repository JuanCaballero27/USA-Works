% !TEX root = ../main.tex

\documentclass[../main.tex]{subfiles}
\begin{document}
\definecolor{qqccqq}{rgb}{0,0.8,0}
\definecolor{qqqqff}{rgb}{0,0,1}
\question \textbf{Ejercicio 1:} Derivar cada una de las siguientes funciones:
\begin{partes}
    \parte $f(x) = \log_{(e^x)} (\sin x)$\\
    Por propiedades de los logaritmos, la función puede ser reescrita como:
    \begin{align*}
        f(x) &= \frac{\log(\sin x)}{\log(e^x)}\\
        &= \frac{\log(\sin x)}{x}\\
        &= x^{-1} \cdot \log(\sin x)
    \end{align*}
    Y aplicando la regla de la derivada de un producto:
    \begin{align*}
        \frac{d}{dx} f(x) &= x^{-1} \cdot \frac{d}{dx} \log(\sin x) + \log(\sin x) \cdot \frac{d}{dx} x^{-1}\\
        &= x^{-1} \cdot \frac{1}{\sin(x)}\cdot \cos(x) + \log(\sin x) \cdot \left(-\frac{1}{x^2}\right)\\
        &= \frac{1}{x} \cdot \frac{\cos x}{\sin x} - \frac{\log(\sin x)}{x^2}\\
        &= \frac{\cot x}{x} - \frac{\log(\sin x)}{x^2}\\
        &= \frac{x\cdot \cot x - \log(\sin x)}{x^2}
    \end{align*}

    \parte $f(x) = (\log x)^{\log x}$\\
    Por definición de exponenciación, podemos escribir la función como:
    \begin{align*}
        f(x) &= e^{\log x \cdot \log(\log x)}
    \end{align*}
\end{partes}

\question \textbf{Ejercicio 2:} Aplicar derivación logaritmica para para obtener la derivada de cada una de las siguientes funciones:
\begin{partes}
    \parte $f(x) = \sin x^{\cos x} + \cos x^{\sin x}$\\

    Integrando cada sumando:
    \begin{align*}
        \frac{d}{dx} \sin x^{\cos x} &= \frac{d}{dx} e^{\log(\sin x){\cos x}}\\
        &= e^{\log(\sin x) \cos x} \cdot \left[\frac{\cos^2 x}{\sin x} - \sin x \log(\sin x)\right]\\ 
        &= \sin x^{\cos x} \cdot \left[\frac{\cos^2 x}{\sin x} - \sin x \log(\sin x)\right] 
    \end{align*}

    Y el otro:
    \begin{align*}
        \frac{d}{dx} \cos x^{\sin x} &= \frac{d}{dx} e^{\log(\cos x){\sin x}}\\
        &= e^{\log(\cos x) \sin x} \cdot \left[\cos x \log(\cos x) \frac{\sin^2 x}{\cos x}\right]\\ 
        &= \cos x^{\sin x} \cdot \left[\cos x \log(\cos x) \frac{\sin^2 x}{\cos x}\right]
    \end{align*}

    Por lo que:
    \begin{align*}
        \frac{d}{dx} f(x) &= \sin x^{\cos x} \cdot \left[\frac{\cos^2 x}{\sin x} - \sin x \log(\sin x)\right]  + \cos x^{\sin x} \cdot \left[\cos x \log(\cos x) \frac{\sin^2 x}{\cos x}\right]
    \end{align*}
\end{partes}

\question \textbf{Ejercicio 5:} Hallar los siguientes limites mediante la regla de L'Hopital.
\begin{partes}
    \parte $\lim\limits_{x \to 0} \frac{e^x-1-x-\frac{x^2}{2}}{x^3}$\\
    Aplicando repetidamente la regla de L'Hopital:
    \begin{align*}
        \lim\limits_{x \to 0} \frac{e^x-1-x-\frac{x^2}{2}}{x^3} &= \lim\limits_{x \to 0} \frac{e^x-1-x}{3x^2}\\
        &= \lim\limits_{x \to 0} \frac{e^x-1}{6x}\\
        &= \lim\limits_{x \to 0} \frac{e^x}{6}\\
        &= \frac{e^0}{6}\\
        &= \frac{1}{6}
    \end{align*}

    \parte $\lim\limits_{x \to 0} \frac{\log(1+x) - x + \frac{x^2}{2} - \frac{x^3}{3}}{x^3}$
    Aplicando repetidamente la regla de L'Hopital:
    \begin{align*}
        \lim\limits_{x \to 0} \frac{\log(1+x) - x + \frac{x^2}{2} - \frac{x^3}{3}}{x^3} &=\lim\limits_{x \to 0}  \frac{\log(1+x) - x + \frac{x^2}{2}}{x^3} - \lim\limits_{x \to 0} \frac{\frac{x^3}{3}}{x^3}\\
        &= \lim\limits_{x \to 0} \frac{\frac{1}{x+1} - 1 + 2x}{3x^2} - \frac{1}{3}\\
        &= \lim\limits_{x \to 0} \frac{-\frac{1}{x^2+2x+1} + 2}{6x} - \frac{1}{3}\\
        &= \lim\limits_{x \to 0} \frac{\frac{2}{(x+1)^3}}{6} - \frac{1}{3}\\
        &= \frac{\frac{2}{1}}{6} - \frac{1}{3}\\
        &= 0
    \end{align*}
\end{partes}

\question \textbf{Ejercicio 7:} Demostrar que:
\begin{partes}
    \parte $\cosh(x+y) = \cosh x\cosh y + \sinh x \sinh y$
    \begin{align*}
        \cosh x\cosh y + \sinh x \sinh y &= \frac{e^x+e^{-x}}{2} \cdot \frac{e^y+e^{-y}}{2} + \frac{e^x - e^{-x}}{2} \cdot \frac{e^y-e^{-y}}{2}\\
        &= \frac{(e^x+e^{-x})(e^y+e^{-y}) + (e^x-e^{-x})(e^y-e^{-y})}{4}\\
        &= \frac{e^{x+y} + e^{y-x} + e^{x-y} + e^{-x-y} + e^{x+y} - e^{y-x} - e^{x-y} + e^{-x-y}}{4}\\
        &= \frac{2e^{x+y} + 2e^{-x-y}}{4}\\
        &= \frac{e^{x+y} + e^{-x-y}}{2}\\
        &= \cosh(x+y)
    \end{align*}
\end{partes}

\question \textbf{Ejercicio 8:} Las funciones $\sinh$ y $\tanh$ son inyectivas y por tanto poseen su respectiva función inversa. Pero para la función $\cosh$ no es el caso. Si se restringe su dominio a $[0, \infty)$ tiene una inversa designada por $\text{arccosh}$ definida sobre $[1, \infty)$. Demostrar:
\begin{partes}
    \parte $\frac{d}{dx} \cosh^{-1} = \frac{1}{\sqrt{x^2-1}}$ para $x > 1$\\
    Recordemos que por propiedades de las derivadas para una función $f$ y su inversa $f^{-1}$ tendremos que:
    \begin{align*}
        \frac{d}{dx} f^{-1}(x) &= \frac{1}{\frac{d}{dx} f(x) \circ f^{-1}(x)}
    \end{align*}
    Y recordando que $\frac{d}{dx} \cosh(x) = \senh(x)$ tendremos que:
    \begin{align*}
        \frac{d}{dx} \cosh^{-1}(x) &= \frac{1}{\senh(x) \circ \cosh^{-1}(x)}
    \end{align*}
    además usando la identidad $\cosh^2 - \sinh^2 = 1$ y la definición de función inversa:
    \begin{align*}
        \frac{d}{dx} \cosh^{-1}(x) &= \frac{1}{\sqrt{\cosh^2(\cosh^{-1}(x)) - 1}}\\
        &= \frac{1}{\sqrt{\cosh(\cosh^{-1}(x)) \cdot \cosh(\cosh^{-1}(x)) - 1}}\\
        &=  \frac{1}{\sqrt{x \cdot x - 1}}\\
        &= \frac{1}{\sqrt{x^2-1}}
    \end{align*}
\end{partes}

\question \textbf{Ejercicio 9:} Hallar una formula explicita para $\sinh^{-1}$, $\cosh^{-1}$ y $\tanh^{-1}$.\\

Para $\sinh^{-1}$ basta simplemente con despejar en la formula dada:
\begin{align*}
    y &= \frac{e^{x} - e^{-x}}{2}\\
    2y &= e^{x}-e^{-x}\\
    e^{x} - 2y - e^{-x} &= 0\\
    e^{2x} - 2ye^{x} - 1 &= 0
\end{align*}
Luego, usando la formula cuadratica(Teniendo en cuenta que queremos determinar $e^x$):
\begin{align*}
    e^x &= \frac{-(-2y) \pm \sqrt{4y^2 + 4}}{2}\\
    e^x &= \frac{2y \pm 2\sqrt{y^2 + 1}}{2}\\
    e^x &= y \pm \sqrt{y^2+1}
\end{align*}
Como $e^x$ siempre será un número positivo, podemos tomar la raíz con $+$ y aplicando logaritmo:
\begin{align*}
    e^x &= y + \sqrt{y^2 + 1}\\
    \log(e^x) &= \log(y + \sqrt{y^2+1})\\
    x &= \log(y + \sqrt{y^2+1})
\end{align*}
Por lo que $\sinh^{-1}(x) = \log(x + \sqrt{x^2+1})$. Para $\cosh^{-1}$ es necesario recordar que se hace una restricción del dominio de $\cosh^{-1}$ a $[0, \infty)$ y el dominio de $\cosh^{-1}$ será $[1, \infty)$:
\begin{align*}
    y &= \frac{e^x+e^{-x}}{2}\\
    2y &= e^x+e^{-x}\\
    e^x - 2y + e^{-x} &= 0\\
    e^{2x} - 2ye^x + 1 &= 0
\end{align*}
De nuevo, aplicaremos la formula cuadratica para determinar $e^x$:
\begin{align*}
    e^x &= \frac{-(-2y) \pm \sqrt{4y^2 - 4}}{2}\\
    e^x &= \frac{2y \pm 2\sqrt{y^2-1}}{2}\\
    e^x &= y \pm \sqrt{y^2-1}
\end{align*}
Para ajustar el resultado a la restricción hecha, tendremos que escoger la raíz con $+$ y aplicando logaritmo:
\begin{align*}
    e^x &= y + \sqrt{y^2-1}\\
    \log(e^x) &= \log(y + \sqrt{y^2-1})\\
    x &= \log(y + \sqrt{y^2-1})
\end{align*}
Por lo que $\cosh^{-1}(x) = \log(x + \sqrt{x^2-1})$. Por último, para $\tanh^{-1}$ tendremos:
\begin{align*}
    y &= \frac{e^x - e^{-x}}{e^x + e^{-x}}\\
    y &= \frac{e^{2x} - 1}{e^{2x} + 1}\\
    y(e^{2x} + 1) &= e^{2x}-1\\
    ye^{2x} + y - e^{2x} + 1 &= 0\\
    e^{2x}(y-1) + (y+1) &= 0\\
    e^{2x} &= -\frac{y+1}{y-1}\\
    \log(e^{2x}) &= \log\left(\frac{y+1}{1-y}\right)\\
    2x &= \log(y+1) - \log(1-y)\\
    x &= \frac{\log(y+1)-\log(1-y)}{2}
\end{align*}
Por lo que $\tanh^{-1}(x) = \frac{\log(x+1)-\log(1-x)}{2}$.

\question \textbf{Ejercicio 10:} Demostrar que:
$$F(x) = \int_2^x \frac{1}{\log(t)} dt$$
no es acotada en $[2, \infty)$.
\begin{proof}
    Recordemos que:
    \begin{align*}
        \log(t) &< t\\
        \frac{1}{\log(t)} &> \frac{1}{t}\\
        \int_{2}^x \frac{1}{\log(t)} dt &> \int_{2}^x \frac{1}{t} dt\\
        \int_{2}^x \frac{1}{\log(t)} dt &> \log(x)-\log(2)
    \end{align*}
    Y dado que $\log(x)-\log(2)$ no es acotado en $[2, \infty)$ obviamente no lo será $\int_{2}^x \frac{1}{\log(t)} dt = F(x)$
\end{proof}

\question \textbf{Ejercicio 25:} Dada una función derivable $f$, hallar todas las funciones continuas $g$ que satisfacen:
\begin{align*}
    \int_0^{f(x)} fg &= g(f(x)) - 1
\end{align*}


\end{document}
