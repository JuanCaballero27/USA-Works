% !TEX root = ../main.tex

\documentclass[../../main.tex]{subfiles}
\begin{document}
\definecolor{qqccqq}{rgb}{0,0.8,0}
\definecolor{qqqqff}{rgb}{0,0,1}
\question \textbf{Ejercicio 1:} Hallar los polinomios de Taylor del grado indicado y en el punto indicado para las siguientes funciones:
\begin{partes}
    \parte $f(x) = e^{\sin x}$ de grado 3 en 0\\

    Primero, determinemos las primeras tres derivadas:
    \begin{align*}
        f(x) &= e^{\sin x}\\
        \frac{d}{dx} f(x) &= e^{\sin x} \cos x\\
        \frac{d^2}{dx^2} f(x) &= e^{\sin x}\cos^2 x - \sin x e^{\sin x}\\
        \frac{d^3}{dx^3} f(x) &= e^{\sin x}\cos^3 x - 2\cos x \sin x - e^{\sin x}\sin x \cos x - \cos x e^{\sin x}
    \end{align*}
    Luego, evaluando en $0$ tendremos:
    \begin{align*}
        f(0) &= 1\\
        \frac{d}{dx} f(0) &= 1\\
        \frac{d^2}{dx^2} f(0) &= e^{\sin 0}\cos^2 0 - \sin 0 e^{\sin 0} = 1\\
        \frac{d^3}{dx^3} f(0) &= e^{\sin 0}\cdot \cos^3 0 - 2\cos 0 \cdot \sin 0 - e^{\sin 0}\cdot \sin 0 \cdot \cos 0 - \cos 0 \cdot e^{\sin 0} = 0
    \end{align*}
    Y evaluando el polinomio de Taylor:
    \begin{align*}
        P_{3, 0}(x) &= \sum_{i = 0}^3 \frac{\frac{d^i}{dx^i} f(0)}{i!} x^i\\
        &= 1 + x + \frac{x^2}{2}
    \end{align*}

    \parte $f(x) = \log(x)$ de grado $n$ en $2$\\

    Para empezar, notemos un pequeño patrón dentro de las derivadas de $f$:
    \begin{align*}
        f(x) &= \log(x)\\
        \frac{d}{dx} f(x) &= x^{-1}\\
        \frac{d^2}{dx^2} f(x) &= -x^{-2}\\
        \frac{d^3}{dx^3} f(x) &= 2x^{-3}\\
        \frac{d^4}{dx^4} f(x) &= -6x^{-4}\\
        \dots & \dots\\
        \frac{d^n}{dx^n} f(x) &= (-1)^{n+1} (n-1)! x^{-n}
    \end{align*}
    Con esto, para $n \ge 1$ podremos hacer una generalización:
    \begin{align*}
        P_{n, 2} &= \log(2) + \sum_{i = 1}^n \frac{\frac{d^i}{dx^i} f(2)}{i!} (x-2)^i\\
        &= \log(2) + \sum_{i = 1}^n \frac{(-1)^{i+1} (i-1)! 2^{-i}}{i!} (x-2)^i\\
        &= \log(2) + \sum_{i = 1}^n \frac{(-1)^{i+1} 2^{-i}}{i} (x-2)^i\\
        &= \log(2) + \sum_{i = 1}^n \frac{(-1)^{i+1}}{i\cdot 2^i} (x-2)^i\\
        &= \log(2) + \frac{x-2}{2} - \frac{(x-2)^2}{8} + \frac{(x-2)^3}{24} - \dots + \frac{(-1)^{n+1}}{n\cdot 2^n} (x-2)^n
    \end{align*}

    \parte $f(x) = \frac{1}{1+x^2}$ de grado $2n+1$ en 0\\


\end{partes}

\question Escribir cada polinomio de $x$ en terminos de un polinomio de $x-3$:
\begin{partes}
    \parte $f(x) = x^4-12x^3+44x^2+2x+1$\\

    Bastará cálcular el polinomio de Taylor de grado $4$ en $3$ para $f$ ya que será igual hasta grado 4 a $f$, por lo que primero deberemos derivar la función 4 veces:
    \begin{align*}
        f(x) &= x^4-12x^3+44x^2+2x+1\\
        \frac{d}{dx} f(x) &= 4x^3 -36x^2+88x+2\\
        \frac{d^2}{dx^2} f(x) &= 12x^2 - 72x + 88\\
        \frac{d^3}{dx^3} f(x) &= 24x - 72\\
        \frac{d^4}{dx^4} f(x) &= 24
    \end{align*}
    Y ahora si se evalua en $3$ tendremos:
    \begin{align*}
        f(3) &= 3^4-12\cdot 3^3+44\cdot 3^2+2\cdot 3+1 = 160 \\
        \frac{d}{dx} f(3) &= 4\cdot 3^3 -36\cdot 3^2+88\cdot 3+2 = 50 \\
        \frac{d^2}{dx^2} f(3) &= 12\cdot 3^2 - 72\cdot 3 + 88 = -20\\
        \frac{d^3}{dx^3} f(3) &= 24x - 72 = 0\\
        \frac{d^4}{dx^4} f(3) &= 24 = 24
    \end{align*}
    Por lo que el polinomio quedará:
    \begin{align*}
        f(x) &= \frac{24}{4!}(x-3)^4 + \frac{0}{3!}(x-3)^3 -\frac{20}{2!}(x-3)^2 + \frac{50}{1!}(x-3)^1 + 160\\
        &= (x-3)^4 - 10(x-3)^2+50(x-3)+160
    \end{align*}
\end{partes}
\end{document}