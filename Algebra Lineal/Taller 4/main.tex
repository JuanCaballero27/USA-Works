\documentclass[12pt,a4paper,oneside]{memoir}
\usepackage[top=1cm,left=1cm,right=1.5cm,bottom=2cm]{geometry}
\usepackage[spanish]{babel}
\usepackage[utf8]{inputenc}
\usepackage[colorlinks=true,urlcolor=magenta,citecolor=red,linkcolor=violet,bookmarks=true]{hyperref}
\usepackage[sort&compress,round,comma,authoryear]{natbib}
\usepackage{makeidx}
\usepackage{lscape}
\usepackage{pdflscape}
\usepackage{epstopdf}
\usepackage{booktabs}
\usepackage{pdfpages}
\usepackage{textcomp}
\usepackage[many]{tcolorbox}
\usepackage{empheq}
\usepackage{tasks}
\usepackage{array}
\usepackage{tikz}
\usepackage[T1]{fontenc}
\usepackage{ae}
\usepackage{lipsum}
\usepackage{indentfirst}
\usepackage{graphicx}
\usepackage{subfig}
\usepackage{float}
\usepackage{blindtext}
\usepackage{tabularx}
\usepackage{ragged2e}
\usepackage{xcolor}
\usepackage{multirow}
\usepackage{bookmark}
\usepackage{xcolor}
\usepackage{amsmath,amssymb,amsthm}
\usepackage{lastpage}
\usepackage{epigraph}
\usepackage{enumerate}
\usepackage{enumitem}

\newlist{questions}{enumerate}{3}
\setlist[questions]{label=\arabic*.}
\newcommand{\question}{\item}
\setlist[enumerate,1]{% (
leftmargin=*, itemsep=12pt, label={\textbf{\arabic*.)}}}

\newlist{partes}{enumerate}{3}
\setlist[partes]{label=(\alph*)}
\newcommand{\parte}{\item}
%---
\newlist{subpartes}{enumerate}{3}
\setlist[subpartes]{label=\roman*)}
\newcommand{\subparte}{\item}


\newcommand*\circled[1]{\tikz[baseline=(char.base)]{\node[shape=circle,draw,inner sep=2pt] (char) {#1};}}

\pagestyle{plain}
\newcommand{\instituto}{Universidad Sergio Arboleda}
\newcommand{\curso}{Algebra Lineal 1}
\newcommand{\professor}{Alejandro Garzón Leyton}
\newcommand{\disciplina}{Matemáticas}
\newcommand{\titulo}{Taller 10}
\newcommand{\alumnoI}{Juan Sebastián Caballero Bernal}
\newcommand{\alumnoII}{Luz Ángela Orjuela Nieto}
\newcommand{\X}{\mathbf{X}}
\newcommand{\x}{\mathbf{x}}
\newcommand{\Z}{\mathbf{Z}}
\newcommand{\z}{\mathbf{z}}
\newcommand{\y}{\boldsymbol{y}}
\newcommand{\balpha}{\mbox{${ \bm \alpha}$}}
\newcommand{\bmu}{\mbox{${\bm \mu}$}}
\newcommand{\bbeta}{\mbox{${\bm \beta}$}}
\newcommand{\bteta}{\mbox{${\bm \theta}$}}
\newcommand{\bgama}{\mbox{${\bm \gamma}$}}
\newcommand{\bxi}{\mbox{${\bm \xl}$}}
\newcommand{\bvarphi}{\mbox{${ \bm \varphi}$}}
\newcommand{\SZ}{\mbox{ $Z$}}
\newcommand{\muz}{\mu_{z,l}}
\newcommand{\muo}{\mu_{0,l}}
\newcommand{\etao}{\eta_{0,l}}
\newcommand{\etaz}{\eta_{z,l}}
\newcommand{\xbeta}{x_{l}\bgama}
\newcommand{\mui}{\mu_{l}}
\newcommand{\zetaind}{\zeta \mathtt{I}_{\{s_{l} \in Z \}}} 
\newcommand{\spz}{ s_{l} \in z}
\newcommand{\snpz}{ s_{l} \notin z}
\newcommand{\sps}{ s_{l} \in S}
\newcommand{\gphi}{ \Gamma(\phi)}
\newcommand{\scan}{ \Lambda_{z}}
\newcommand{\gmuop}{ \Gamma(\muo\phi)}
\newcommand{\gmuzp}{ \Gamma(\muz \phi)}
\newcommand{\gumuop}{ \Gamma((1-\muo)\phi)}
\newcommand{\gumuzp}{ \Gamma((1-\muz)\phi)}
\newcommand{\dlobeta}{ \frac{\parteial l_{0}(\bgama, \phi, 0) }{\parteial \bgama}}
\newcommand{\lz}{  l_{z}(\bgama, \phi, \tau)}
\newcommand{\lo}{  l_{0}(\beta, \phi, 0)}
\newcommand{\E}{\mathbb{E}}
\newcommand{\dis}{\displaystyle}
\linespread{1.5}

\newtheorem*{definition*}{Definición}
\newtheorem*{theorem*}{Teorema}
\newtheorem*{axiom*}{Postulado}
\newtheorem{theorem}{Teorema}[section]
\renewcommand*{\proofname}{\textbf{Demostración}}

\begin{document}
%%%%%%%%%%%%%%%%%%%%%%%%%%%%%%%%%%%%%%%%%%%%%%%%%%%%%%%%
%                      Emcabezado                     %
%%%%%%%%%%%%%%%%%%%%%%%%%%%%%%%%%%%%%%%%%%%%%%%%%%%%%%%%
\begin{table}[H]
\centering
\begin{tabular*}{\textwidth}{l@{\extracolsep{\fill}}l@{\extracolsep{\fill}}}
    \begin{tabular}[l]{@{}l@{}}
        \textbf{\instituto}\\
        \textbf{Disciplina: \disciplina}\\
        \textbf{Profesor: \professor}\\ 
    \end{tabular} & 
    \begin{tabular}[l]{@{}l@{}}
        {\curso}\\
        {\alumnoI}\\
        {\alumnoII}\\
    \end{tabular}
\end{tabular*}
\end{table}
\begin{center}
\rule[2ex]{\textwidth}{1pt}
{\Large{\titulo}}
\end{center}
\rule[2ex]{\textwidth}{1pt}
\begin{questions}[label=\protect\circled{\bfseries\arabic*}]
%%%%%%%%%%%%%%%%%%%%%%%%%%%%%%%%%%%%%%%%%%%%%%%%%%%%%%%%
%                       Preguntas                      %
%%%%%%%%%%%%%%%%%%%%%%%%%%%%%%%%%%%%%%%%%%%%%%%%%%%%%%%%
\question \textbf{Pregunta 12:} Resolver el siguiente sistema de ecuaciones lineales

$$\begin{matrix}
    x_1 &-& 3x_2 &+& 4x_3 &= -4\\
    3x_1 &-& 7x_2 &+& 7x_3 &= -8\\
    -4x_1 &+& 6x_2 &-& x_3 &= 1\\
\end{matrix}$$

Para resolver el sistema, representamos los coeficientes y el termino independiente en una matriz como sigue:\\
$$\begin{bmatrix}1 & -3 & 4 & -4\\3 & -7 & 7 & -8\\-4 & 6 & -1 & 7\\\end{bmatrix}$$
Ahora, se busca dejar en la forma escalonada reducida a la matríz, por lo que se efectuan las siguientes operaciones:

\begin{center}
\begin{tabular}{cc}
    \vspace{0.5cm}
    $\begin{bmatrix}1 & -3 & 4 & -4\\3 & -7 & 7 & -8\\-4 & 6 & -1 & 7\\\end{bmatrix}$ & Matriz original \\
    \vspace{0.5cm}
    $\begin{bmatrix} 1 & -3 & 4 & -4\\ 3 & -7 & 7 & -8 \\ 4 & -6 & 1 & -7\end{bmatrix}$ & $F_3 \leftarrow -F_3$\\
    \vspace{0.5cm}
    $\begin{bmatrix}1 & -3 & 4 & -4\\ 3 & -7 & 7 & -8\\ 0 & 6 & -15 & 9 \end{bmatrix}$ & $F_3 \leftarrow F_3 - 4F_1$\\
        \vspace{0.5cm}

    $\begin{bmatrix}1 & -3 & 4 & -4\\ 3 & -7 & 7 & -8\\ 0 & 2 & -5 & 3 \end{bmatrix}$ & $F_3 \leftarrow \frac{1}{3}F_3$\\
        \vspace{0.5cm}

    $\begin{bmatrix}1 & -3 & 4 & -4\\ 0 & 2 & -5 & 4\\ 0 & 2 & -5 & 3 \end{bmatrix}$ & $F_2 \leftarrow F_2 - 3F_1$\\
        \vspace{0.5cm}

    $\begin{bmatrix}1 & -3 & 4 & -4\\ 0 & 2 & -5 & 4\\ 0 & 0 & 0 & 1 \end{bmatrix}$ & $F_3 \leftarrow F_3 - F_2$
\end{tabular}
\end{center}
Luego, volviendo a escribir las ecuaciones en base a la última matriz resultante:
$$\begin{matrix}
    x_1 &-& 3x_2 &+& 4x_3 &= -4\\
    0x_1 &+& 2x_2 &-& 5x_3 &= 4\\
    0x_1 &+& 0x_2 &-& 0x_3 &= 1\\
\end{matrix}$$
Y dado que para todo $\alpha, \beta, \gamma \in \mathbb{C}$ nunca es verdad que:
$$\alpha \cdot 0 + \beta \cdot 0 + \gamma \cdot 0 = 1$$
Podemos concluir que el sistema es inconsistente. Y gracias a que es equivalente al sistema original, podemos concluir que el sistema original también es inconsistente.

\question \textbf{Pregunta 20:} Determinar los valores de $h$ para los cuales la matriz es la matriz aumentada de un sistema linear consistente.
$$\begin{bmatrix}
1 & h & -3\\
-2 & 4 & 6\\
\end{bmatrix}$$

\text{Para determinar lo requerido en la pregunta, basta con preguntarse cuando el sistema de ecucaciones 
lineales de la matriz dada no sería consistente. Es decir, en cuales podremos obtener ecuaciones de 
solución vacía o rectas paralelas.\\ 
Dado que para ningún valor se genera una ecuación del tipo $0x_1 + 0x_2 = -3$ y gracias a que al 
despejar el sistema es imposible generar rectas paralelas con las ecuaciones, dicho sistema tiene 
solución para $h \in \mathbb{C}$.}


\question \textbf{Pregunta 30:} Determinar la operación elemental de filas que transforma la primera matriz en la segunda, y determinar la operación inversa que transforma la segunda en la primera.
\begin{center}
    \begin{tabular}{c c}
        $\begin{bmatrix} 1 & 3 & -4\\ 0 & -2 & 6\\ 0 & -5 & 9 \end{bmatrix}$ & $\begin{bmatrix} 1 & 3 & -4\\ 0 & 1 & -3\\ 0 & -5 & 9 \end{bmatrix}$ \\
    \end{tabular}
\end{center}
Para empezar, se determina cual fila de la primera matriz es aquella que es alterada para obtener la segunda matriz. En este caso, $F_2$ es aquella fila que es alterada, y como se puede apreciar, el cambio se hace al multiplicar por $-\frac{1}{2}$. Para comprobar:
\begin{center}
    \begin{tabular}{c c}
        \vspace{0.5cm}
        $\begin{bmatrix} 1 & 3 & -4\\ 0 & -2 & 6\\ 0 & -5 & 9 \end{bmatrix}$ & Matriz Original \\
        $\begin{bmatrix} 1 & 3 & -4\\ 0 & 1 & -3\\ 0 & -5 & 9 \end{bmatrix}$ & $F_2 \leftarrow -\frac{1}{2} F_2$
    \end{tabular}
\end{center}
A su vez, eso nos permite determinar que la operación inversa que convierte la segunda matriz en la primera será multiplicar por $-2$ la segunda fila. Resultando así:
\begin{center}
    \begin{tabular}{c c}
        \vspace{0.5cm}
        $\begin{bmatrix} 1 & 3 & -4\\ 0 & 1 & -3\\ 0 & -5 & 9 \end{bmatrix}$ & Matriz Original \\
        $\begin{bmatrix} 1 & 3 & -4\\ 0 & -2 & 6\\ 0 & -5 & 9 \end{bmatrix}$ & $F_2 \leftarrow -2 F_2$
    \end{tabular}
\end{center}
\end{questions}
\end{document}