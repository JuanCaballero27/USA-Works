\documentclass[12pt,a4paper,oneside]{memoir}
\usepackage{pstricks-add}
\usepackage[top=1cm,left=1cm,right=1.5cm,bottom=2cm]{geometry}
\usepackage[spanish]{babel}
\usepackage[utf8]{inputenc}
\usepackage[colorlinks=true,urlcolor=magenta,citecolor=red,linkcolor=violet,bookmarks=true]{hyperref}
\usepackage[sort&compress,round,comma,authoryear]{natbib}
\usepackage{makeidx}
\usepackage{lscape}
\usepackage{pdflscape}
\usepackage{epstopdf}
\usepackage{booktabs}
\usepackage{pdfpages}
\usepackage{textcomp}
\usepackage[many]{tcolorbox}
\usepackage{empheq}
\usepackage{tasks}
\usepackage{array}
\usepackage{tikz}
\usepackage[T1]{fontenc}
\usepackage{ae}
\usepackage{lipsum}
\usepackage{indentfirst}
\usepackage{graphicx}
\usepackage{subfig}
\usepackage{float}
\usepackage{blindtext}
\usepackage{tabularx}
\usepackage{ragged2e}
\usepackage{xcolor}
\usepackage{multirow}
\usepackage{bookmark}
\usepackage{pgfplots}
\usepackage{amsmath,amssymb,amsthm}
\usepackage{lastpage}
\usepackage{epigraph}
\usepackage{enumerate}
\usepackage{enumitem}
\usepackage{mathrsfs}
\usepackage{tikz}
\usepackage{pgfplots}
\pgfplotsset{compat=1.15}

\usetikzlibrary{arrows}
\usepackage{subfiles} % Insert the commands.tex file which contains the majority of the structure behind the template
\pgfplotsset{compat=1.15}

\newlist{questions}{enumerate}{3}
\setlist[questions]{label=\arabic*.}
\newcommand{\question}{\item}
\setlist[enumerate,1]{% (
leftmargin=*, itemsep=12pt, label={\textbf{\arabic*.)}}}

\newlist{partes}{enumerate}{3}
\setlist[partes]{label=(\alph*)}
\newcommand{\parte}{\item}
%---
\newlist{subpartes}{enumerate}{3}
\setlist[subpartes]{label=\roman*)}
\newcommand{\subparte}{\item}


\newcommand*\circled[1]{\tikz[baseline=(char.base)]{\node[shape=circle,draw,inner sep=2pt] (char) {#1};}}

\pagestyle{plain}
\newcommand{\instituto}{Universidad Sergio Arboleda}
\newcommand{\curso}{Matemáticas Discretas}
\newcommand{\professor}{Diego Villamizar}
\newcommand{\disciplina}{Matemáticas}
\newcommand{\titulo}{Taller 5}
\newcommand{\alumnoI}{Juan Sebastián Caballero Bernal}
\newcommand{\alumnoII}{Luz Ángela Orjuela Nieto}
\linespread{1.5}
\pagestyle{empty}
\newtheorem*{definition*}{Definición}
\newtheorem*{theorem*}{Teorema}
\newtheorem*{axiom*}{Postulado}
\newtheorem{theorem}{Teorema}[section]
\renewcommand*{\proofname}{\textbf{Demostración}}

\begin{document}
%%%%%%%%%%%%%%%%%%%%%%%%%%%%%%%%%%%%%%%%%%%%%%%%%%%%%%%%
%                      Emcabezado                     %
%%%%%%%%%%%%%%%%%%%%%%%%%%%%%%%%%%%%%%%%%%%%%%%%%%%%%%%%
\begin{table}[H]
    \centering
    \begin{tabular*}{\textwidth}{l@{\extracolsep{\fill}}l@{\extracolsep{\fill}}}
        \begin{tabular}[l]{@{}l@{}}
            \textbf{\instituto}              \\
            \textbf{Disciplina: \disciplina} \\
            \textbf{Profesor: \professor}    \\
        \end{tabular} &
        \begin{tabular}[l]{@{}l@{}}
            {\curso}   \\
            {\alumnoI} \\
        \end{tabular}
    \end{tabular*}
\end{table}
\begin{center}
    \rule[2ex]{\textwidth}{1pt}
    {\Large{\titulo}}
\end{center}
\rule[2ex]{\textwidth}{1pt}
\begin{questions}[label=\protect\circled{\bfseries\arabic*}]
    \question Demuestre que el número de caminatas de $v_1$ a $v_2$ en el grafo
    $$G = (\{v_1, v_2\}, \{(v_1, v_1), (v_1, v_2), (v_2, v_1)\})$$
    de longitud $n$ es $F_n$.

    \begin{proof}
        Denote como $L_n$ el conjunto de caminatas de $v_1$ a $v_2$ de longitud $n$. Para $n = 0$ no existe ningún camino de $v_1$ a $v_2$ de longitud $0$ ya que dicho caminata sería $\{v_1\}$ lo cúal nunca llega a $v_2$. Luego, $|L_0| = 0$. Además, para $n = 1$ el único caminata posible es $\{v_1, v_2\}$, ya que cualquier otra caminata tendría más elementos, por lo que $|L_1| = 1$. Con esto, ya tenemos que el caso base de $F_n$ y de $|L_n|$ son iguales. Solo es necesario probar que para $n \ge 2$ se tiene que $F_n = |L_n|$. Para ello, considere la siguiente biyección:
        $$\begin{matrix}
            \varphi: & L_{n} \cup L_{n+1} & \to & L_{n+2}\\
            & C = \{v_{i}\} & \mapsto & \varphi(C)
        \end{matrix}$$
        Donde si $C \in L_{n}$ entonces $\varphi(C) = \{u_i\}_{i \in [n+3]}$ donde $u_1 = 1$, $u_2 = 2$ y $u_i = v_{i-2}$ para todo $i \ge 3$, y si $C \in L_{n+1}$ entonces $\varphi(C) = \{w_i\}_{i \in [n+3]}$ de forma que $w_1 = 1$ y $w_i = v_{i-1}$ para todo $i \ge 2$.
        \begin{itemize}
            \item \textbf{Inyectividad:} Note que no pueden existir elementos que pertenezcan a $L_n$ y $L_{n+1}$ al tiempo, ya que eso implicaría que $n = n+1$. Suponga que $C = \{a_i\}_{i \in [n]}$ y $D = \{b_i\}_{i \in [n]}$ son caminatas de $L_n$ de forma que $\varphi(C) = \varphi(D)$. Eso quiere decir que las sucesiones $\{v_i\}_{i \in [n+3]}$ y $\{u_i\}_{i \in [n+3]}$ son iguales(En correspondencia con $\varphi(C)$ y $\varphi(D)$). Luego, quiere decir que $v_{i} = u_i$ para todo $i \in [n+3]$. Ya es fácil determinar $v_1 = u_1 = 1$ y que $v_2 = u_2 = 2$. Suponga que $i \ge 3$ y llame ahora $k = i-2$ y dado que $3 \le i \le n+3$ entonces $1 \le k \le n+1$ y $v_i = u_i$ entonces $a_{k} = b_k$ para todo $k \in [n+1]$, por lo que $C = D$. De manera similar se concluye que si $C, D \in L_{n+1}$ y $\varphi(C) = \varphi(D)$ entonces $C = D$(Reemplazando $k = i - 1$).
            \item \textbf{Sobreyectividad:} Note que siempre que tome $C = \{v_i\}_{i \in [n+3]}$ una caminata de longitud $n+2$, $v_1 = 1$, pero $v_2$ puede tomar dos valores. Si $v_2 = 1$ entonces definida la caminata $A = \{u_i\}_{i \in [n+2]}$ con $u_i = v_{i+1}$, y esta es una caminata de $v_1$ a $v_2$ de longitud $n+1$, y luego $\varphi(A) = C$. Si $v_2 = 2$ entonces defina $B = \{u_i\}_{i \in [n+1]}$ de forma que $u_i = v_{i + 2}$ y esta será una caminata de $v_1$ a $v_2$ de longitud $n$, para obtener que $\varphi(B) = C$. 
        \end{itemize} 
        Luego, gracias a que demostramos que $L_{n} \cup L_{n+1} \cong L_{n+2}$ y gracias al hecho de que $L_{n} \cap L_{n+1} = \emptyset$ demostramos que $|L_n| + |L_{n+1}|  = |L_{n+2}|$, por lo que al comprobar los casos bases y la recursión de $F_n$ con $L_{n}$, podemos concluir que en general $|L_{n}| = F_n$.
    \end{proof}

    \question Recuerde que dada una relación binaria $R \subseteq X^2$ con $X$ un conjunto finito de la forma $\{x_1, x_2, \dots, x_n\}$ puede
    representarse de forma gráfica como una matriz $A_R$ de la siguiente forma:
    \begin{align*}
        A & =
        \begin{pmatrix}
            a_{1,1} & a_{1,2} & \cdots & a_{1,n} \\
            a_{2,1} & a_{2,2} & \cdots & a_{2,n} \\
            \vdots  & \vdots  & \ddots & \vdots  \\
            a_{n,1} & a_{n,2} & \cdots & a_{n,n}
        \end{pmatrix},
    \end{align*}
    donde:
    \begin{align*}
        a_{i, j} & = \begin{cases}
            1, & (x_i, x_j) \in R    \\
            0, & (x_i, x_j)\not\in R \\
        \end{cases}
    \end{align*}
    \begin{partes}
        \parte ¿Que matriz representa la relación de aristas del grafo G del punto anterior?\\
        La matriz definida por:
        \begin{align*}
            A_G &= \begin{pmatrix}
                1 & 1\\
                1 & 0
            \end{pmatrix}
        \end{align*}
        \parte Si $A$ y $B$ son de la forma:
        \begin{equation*}
            A =
            \begin{pmatrix}
                a_{1,1} & a_{1,2} & \cdots & a_{1,n} \\
                a_{2,1} & a_{2,2} & \cdots & a_{2,n} \\
                \vdots  & \vdots  & \ddots & \vdots  \\
                a_{n,1} & a_{n,2} & \cdots & a_{n,n}
            \end{pmatrix},\, B =
            \begin{pmatrix}
                b_{1,1} & b_{1,2} & \cdots & b_{1,n} \\
                b_{2,1} & b_{2,2} & \cdots & b_{2,n} \\
                \vdots  & \vdots  & \ddots & \vdots  \\
                b_{n,1} & b_{n,2} & \cdots & b_{n,n}
            \end{pmatrix},
        \end{equation*}
        entonces $A\cdot B$ es la matriz
        \begin{equation*}
            A\cdot B =
            \begin{pmatrix}
                c_{1,1} & c_{1,2} & \cdots & c_{1,n} \\
                c_{2,1} & c_{2,2} & \cdots & c_{2,n} \\
                \vdots  & \vdots  & \ddots & \vdots  \\
                c_{n,1} & c_{n,2} & \cdots & c_{n,n}
            \end{pmatrix}
        \end{equation*}
        con $c_{i, j} = \sum_{k = 1}^n a_{i, k} b_{k, j}$. Pruebe usando inducción que si $G = (V, E)$ es un grafo con $E \subseteq V^2$ entonces:
        $$A_E ^n=\underbrace{A_E\cdot A_E\cdots A_E\cdot A_E}_{n\text{ veces}}=\begin{pmatrix}
                a^{(n)}_{1,1} & a^{(n)}_{1,2} & \cdots & a^{(n)}_{1,n} \\
                a^{(n)}_{2,1} & a^{(n)}_{2,2} & \cdots & a^{(n)}_{2,n} \\
                \vdots        & \vdots        & \ddots & \vdots        \\
                a^{(n)}_{n,1} & a^{(n)}_{n,2} & \cdots & a^{(n)}_{n,n}
            \end{pmatrix}$$
        contiene como entradas, $a^{(n)}_{i, j}$ que cuenta el número de caminatas de $v_i$ a $v_j$ de longitud $n$.

        \begin{proof}
            El caso $k = 1$ es trivial, ya que la cantidad de caminatas de $v_i$ a $v_k$ de longitud $1$ será la arista que los conecte o no. Supongamos que en $A_E^{k} = (a^{(k)}_{i, j})$, la entrada $a^{(k)}_{i, j}$ representa la cantidad de caminatas de $v_i$ a $v_j$ de longitud $k$. Luego, para el producto $A_E^{k} \cdot A$ cada entrada estará dada, gracias a la definición del producto de matrices como:
            \begin{align*}
                a^{(k+1)}_{i, j} &= \sum_{l = 1}^k a_{i, l}^{(n)} \cdot a_{l, j}
            \end{align*} 
            Ahora, la entrada $a_{i, l}^{(k)}$ es la cantidad de caminatas desde $v_i$ hasta $v_l$ de longitud $l$. Luego, si hay caminatas de $v_i$ a $v_l$, esta servirá para llegar a $v_j$ solo si $v_l$ esta conectado con $v_j$. En caso de que sí, $a_{l, j}$ será un $1$ contando que por los $a_{i, l}^{(k)}$ caminos se puede llegar a $v_j$, y note que será de longitud $k+1$ ya que el camino anterior era de longitud $k$ y le agregamos un vertice más. En caso de que alguno de los dos valores sea $0$ quiere decir que no se puede llegar a $v_j$ através de $v_l$. Luego, al verificar esto para $1 \le l \le k$ y sumarlo, serán los posibles caminos de $v_i$ a $v_j$ de longitud $k+1$.
        \end{proof}

        \parte Pruebe que si $G$ es el grafo del problema 1, entonces:
        \begin{align*}
            A^n_G & = \begin{pmatrix}
                          F_{n+1} & F_n     \\
                          F_n     & F_{n-1}
                      \end{pmatrix}
        \end{align*}
        \begin{proof}
            Ya determinamos en el punto anterior que la entrada de la matriz anterior son las caminatas de $v_i$ a $v_j$ de longitud $n$. Recordemos por el punto 1, que el número de caminatas de $v_1$ a $v_2$ de longitud $n$ es $F_n$ lo que explica las entradas $a_{1, 2}$ y $a_{2, 1}$(Note que las caminatas de $v_2$ a $v_1$ de longitud $n$ es invertir aquellas de $v_1$ a $v_2$). Para las otras dos entradas de la matriz:
            \begin{itemize}
                \item Para demostrar que $a_{1, 1}^{(n)} = F_{n+1}$, se usará inducción. El caso $n = 1$ es evidente, por lo que directamente supongamos que es verdad para $n$ y demostraremos que pasa lo mismo para $n+1$. Note que esto saldrá directamente de la multiplicación de $A^n \cdot A$, y la entrada $a_{1, 1}^{(n+1)}$ será:
                \begin{align*}
                    a_{1, 1}^{(n+1)} &= \sum_{k = 1}^2 a_{i, k}^{(n)} \cdot a_{k, j}\\
                    &= a_{1, 1}^{(n)} \cdot a_{1, 1} + a_{2, 1}^{(n)} \cdot a_{2, 1}\\
                    &= F_{n+1} \cdot 1 + F_{n} \cdot 1\\
                    &= F_{n+2}
                \end{align*}

                \item Para demostrar que $a_{2, 2}^{(n)} = F_{n-1}$, se usará inducción. El caso $n = 1$ es evidente, por lo que directamente supongamos que es verdad para $n$ y demostraremos que pasa lo mismo para $n+1$. Note que de nuevo, saldrá sobre la multiplicación de $A^n \cdot A$, y la entrada $a_{2, 2}^{(n+1)}$ será:
                \begin{align*}
                    a_{2, 2}^{(n+1)} &= \sum_{k = 1}^2 a_{2, k}^{(n)} \cdot a_{k, 2}\\
                    &= a_{2, 1}^{(n)} \cdot a_{1, 2} + a_{2, 2}^{(n)} \cdot a_{2, 2}\\
                    &= F_{n} \cdot 1 + F_{n-1} \cdot 0\\
                    &= F_{n}
                \end{align*}
            \end{itemize}
        \end{proof}
    \end{partes}



    \question Sea $G$ un grafo no dirigido y conexo. Suponga que $e \in E$ es una arista que está contenida en un cíclo. Pruebe que el grafo $G' = (V, E \setminus \{e\})$ es conexo.

    \begin{proof}
        Dado que $e$ pertenece a un cíclo, quiere decir que existe un ciclo(Como sucesión de vertices) $\{v_i\}_{i \in [k+1]}$ donde para algún $n$, $(v_n, v_{n+1}) = e$. Supongamos que en realidad $G'$ no es convexo, es decir $|V \backslash R_{G'}| = 2$. Esto quiere decir que existen dos clases distintas de equivalencia, $[[u]]$ y $[[v]]$. De esto, podemos derivar dos posibles escenarios:
        \begin{itemize}
            \item $v_n \in [[u]]$ y $v_{n+1} \in [[v]]$, es decir, no existe una caminata desde $v_n$ hasta $v_{n+1}$ en $G'$. Pero gracias a la existencia del ciclo donde pertenece $e$, y gracias a que $G$ es no dirigido, podemos generar dos caminos:
            \begin{align*}
                \{v_n, v_{n-1}, \dots, v_1\} & & \{v_1, v_{k}, \dots v_{n+1}\}
            \end{align*}
            Pero esto quiere decir que existe un camino desde $v_n$ a $v_{n+1}$(Por la transitividad de la relación), concluyendo que $[[u]] \cap [[v]] \neq \emptyset$ lo que contradice el hecho de que sean clases de equivalencia distintas.

            \item Con lo anterior concluimos que sin perdida de generalidad, $v_n, v_{n+1} \in [[u]]$. Ahora, por lo menos $v \in [[v]]$, por lo que no debería existir ninguna caminata de $v_n$ a $v$ en $G'$. Recordemos que dicho caminata si debe existir en el grafo $G$ ya que era conexo, es decir existe $\{a_i\}_{i \in [x+1]}$ una caminata de forma que $a_i = v_n$ y $a_{x+1} = v \neq v_{n+1}$. Note que $a_{2} = v_{n+1}$ ya que si no fuese así, $e$ nunca estará en esta caminata, por lo que será totalmente valido en $G'$, contradiciendo que no hay caminata de $v_n$ a $v$(De manera más general, podemos concluir que el número de veces que $a_i = v_n$ y $a_{i+1} = v_{n+1}$ debe ser mayor que el número de veces que $a_i = v_{n+1}$ y $a_{i+1} = v_n$ ya que si no fuera así, podríamos encontrar una caminata directa desde $v_n$ a $v$, de nuevo, una contradicción). Podemos deducir entonces que existe un caminata de $v_{n+1}$ y $v$, pero esto contradice el hecho de que $[[u]] \cap [[v]] = \emptyset$, ya que existiría una caminata desde $v_{n+1}$ hasta $v$, y por extensión, una caminata de $v_n$ hasta $v$.
        \end{itemize}
        Hemos visto que suponer que existen dos clases de equivalencia distintas genra una contradicción, por lo que podemos concluir que $[[u]] = [[v]]$(De manera inductiva, para más de dos clases de equivalencia distintas, podemos concluir dos a dos que son iguales). Por lo que $ |V \backslash R_{G'}| = 1$ y por tanto $G'$ es conexo.
    \end{proof}


    \question Sea $G$ un grafo no dirigido y sin loops. Si $|V| = n$, $|Deg(x)| = k$ para todo $x \in V$ y:
    \begin{align*}
        \begin{cases}
            k \ge \frac{n-3}{2} & \text{ Si $n-1$ es divisible por 4}\\
            k \ge \frac{n-1}{2} & \text{ Si $n-1$ no es divisible por 4}
        \end{cases}
    \end{align*}
    entonces $G$ es un grafo conexo.
    \begin{proof}
        Suponga que en realidad el grafo no es conexo. Es decir, $|V/R_G| = 2$. Desde este punto es bueno notar que cada componente conexa del grafo tendrá que tener $k+1$ vertices para poder tener grado $k$ en cada vertice. Luego, tendremos dos casos:
        \begin{itemize}
            \item Si $n-1$ es divisible por 4, entonces $n-1$ es par. Luego, esto nos dice que $n$ debería ser impar. Ahora, gracias a que las componentes conexas forman una partición de $V$, la suma de la cantidad de elementos en cada una deberá ser $|V|$. Esto es $k+1+k+1 = 2k+2$, pero note que este es un número par, por lo que $n$ debería ser par, lo que contradice el hecho de que sea impar.
            \item Si $n-1$ no es divisible por 4, la suma de los elementos de las componentes conexas seguirá siendo $2k+2$ que debe ser $n$. Esto nos permite desarrollar la igualdad dada por los casos como sigue:
            \begin{align*}
                k &\ge \frac{n-1}{2}\\
                2k &\ge n-1\\
                2k+2 &\ge n+1\\
                n &\ge n+1\\
                0 &\ge 1
            \end{align*}
            Lo que de nuevo es una contradicción.
        \end{itemize}
        Para extender este razonamiento a varias componentes conexas, basta con pensar en que si hay $s$ componentes conexas entonces el número de vertices será $sk + s$ que será $n$. Eso llevará a una contradicción con $n$ y $n-1$, o con la desigualdad dada por los casos.
    \end{proof}

    Un $r-$coloramiento de un grafo no dirigido $G = (V, E)$ con $E \subseteq \binom{V}{2}$ es una función $f: V \to [r]$ tal que $f(x) \neq f(y)$ si $\{x, y\} \in E$. Un grafo se dice $r-$coloreable si existe un $r-$coloramiento en él.

    \question ¿Qué tipo de grafos son $1-$coloreables?\\
    Para que un grafo sea $1-$coloreable, quiere decir que existe una función $f: V \to [1]$, tal que $f(x) \neq f(y)$ si $\{x, y\} \in E$. Pero note que $f(x) = 1$ para todo $x \in V$, por lo que el contrareciproco de la condición nos diría que \textit{$f(x) = f(y)$ entonces $\{x, y\} \not\in E$}, lo que permite concluir que $\{u, v\} \not\in E$ para todo $u, v \in V$, por lo que $E = \emptyset$ y por tanto los únicos grafos que son $1-$coloreable son los grafos vácios $V_n$.
    
    \question Pruebe que si $T = (V, E)$ es arbol, entonces es $2-$coloreable. ¿Cuantos $2-$coloreamientos del arbol hay?
    \begin{proof}
        Recuerde que para un arbol $T$, existe una función $d: V^2 \to \mathbb{Z}^{\ge 0}$ que representa la distancia entre dos vertices del arbol. Tome cualquier elemento arbitrario $r \in V$, y considere $T_r$. Definiremos una función $f: V \to \{1, 2\}$ de forma que $f(x) = f(y)$ si y solo si $d(r, x) = d(r, y)$. Esta función es un coloreamiento. Supongamos que $f(x) = f(y)$ pero $\{x, y\} \in E$. Nombre $k$ como $d(x, r), d(y, r)$, entonces existen dos caminos $\{v_i\}_{i \in [k+1]}$ y $\{u_i\}_{i \in [k+1]}$, tal que $v_1 = x, u_1 = y, v_{k+1} = u_{k+1}$. Pero se puede hacer un nuevo camino $\{w_i\}_{i \in [k+2]}$ de forma que $w_i = v_{k+1-i+1}$ para $ 1 \le i \le k+1$ y $w_{k+2} = x$. Y al combinar los caminos $\{v_i\}_{i \in [k+1]}$ y $\{w_i\}_{i \in [k+2]}$ formamos un cíclo dentro del arbol, lo cúal es una contradicción con la definición de arbol. Por tanto, si $f(x) = f(y)$, $\{x, y\} \not \in E$, lo que por definición nos muestra que $f$ es un 2-coloreamiento de $T$.

        Note que de manera más general $f(x) = f(y)$ si y solo si $d(r, x) \equiv d(r, y) (\mod 2 \,)$. Moralmente, esto indica que el coloreamiento los "niveles" o profundidad de un "arbol", en base a si la profundidad es par o impar. Luego, solo hay dos formas que intercambiar ambos estados, por lo que existen únicamente 2 $2-$coloreamientos para $T_r$. Pero note que la elección de $r$ no afecta el coloreamiento de $T$, por lo que en general para $T$ existen 2 $2-$coloreamientos.
    \end{proof}

    \question De una expresión en terminos de $n$ y $m$ para el número de $m-$coloreamientos de:
    \begin{itemize}
        \item $V_n$: $\to m^n$
        \item $G = ([n], E)$ con $E = \{(i, i+1): i \in [n-1]\}$: $\to m \cdot (m-1)^{n-1}$
        \item $K_n$: $\to \binom{m}{n} \cdot n! \cdot n!$ 
    \end{itemize}

    \question Note que si un grafo es $k-$coloreable, entonces es $(k+1)-$coloreable. Denote $\chi(G)$ el minimo número $k$ tal que $G$ es $k-$coloreable.
    \begin{partes}
        \parte Cálcule $\chi(G)$ para los grafos del punto anterior.
        \begin{itemize}
            \item $\chi(V_n)$ es $1$ dado que como se demostró en la primera parte del punto, $V_n$ es $1-$coloreable y dado que $1$ es el minimo número natural, $\chi(V_n) = 1$.
            \item $\chi(G)$ con $G = ([n], E)$ con $E = \{(i, i+1): i \in [n-1]\}$ es $2$ para todo $n \ge 2$, ya que no puede ser $1$ coloreable, ya que $(1, 2) \in E$ por lo que $f(1) \neq f(2)$. El coloreamiento será una función tal que $f(x) = f(y)$ si y solo si $f(x) \equiv f(y) (\mod{2})$. Para $n = 1$ será $1$, ya que es directamente $V_1$.
            \item $\chi(K_n)$ es $n$ ya que tomando un $v$ arbitrario en $V$, sabemos que $|Deg(v)| = n-1$, por lo que tendremos que tener $n-1$ valores para los vertices, y añadiendo el propio valor de $v$, necesitaremos $n$ valores distintos. Luego, $\chi(K_n) = n$. 
        \end{itemize}

        \parte Pruebe que $\chi(G) \le \Delta(G) + 1$ si $\Delta(G)$ es el máximo grado de $G$ i.e, $\Delta(G) = \max\limits_{x\in V} |Deg(x)|$
        \begin{proof}
            Suponga que $x \in V$ es tal que $|Deg(x)| = \Delta(G)$, entonces para cada uno de los vertices en $Deg(x)$ se puede escoger un color diferente para cada uno de los vertices(Aunque esto no llega a ser necesario a no ser que estén conectados entre sí), y dado que no hay otro vertice con dicha caracteristica, llegaremos a que agregando la imagen para $x$ en el coloreamiento:
            $$\chi(G) \le \Delta(G) + 1$$
        \end{proof}
    \end{partes}
\end{questions}
\end{document}