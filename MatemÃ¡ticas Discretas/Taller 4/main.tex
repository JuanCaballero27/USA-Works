\documentclass[12pt,a4paper,oneside]{memoir}
\usepackage{pstricks-add}
\usepackage[top=1cm,left=1cm,right=1.5cm,bottom=2cm]{geometry}
\usepackage[spanish]{babel}
\usepackage[utf8]{inputenc}
\usepackage[colorlinks=true,urlcolor=magenta,citecolor=red,linkcolor=violet,bookmarks=true]{hyperref}
\usepackage[sort&compress,round,comma,authoryear]{natbib}
\usepackage{makeidx}
\usepackage{lscape}
\usepackage{pdflscape}
\usepackage{epstopdf}
\usepackage{booktabs}
\usepackage{pdfpages}
\usepackage{textcomp}
\usepackage[many]{tcolorbox}
\usepackage{empheq}
\usepackage{tasks}
\usepackage{array}
\usepackage{tikz}
\usepackage[T1]{fontenc}
\usepackage{ae}
\usepackage{lipsum}
\usepackage{indentfirst}
\usepackage{graphicx}
\usepackage{subfig}
\usepackage{float}
\usepackage{blindtext}
\usepackage{tabularx}
\usepackage{ragged2e}
\usepackage{xcolor}
\usepackage{multirow}
\usepackage{bookmark}
\usepackage{pgfplots}
\usepackage{amsmath,amssymb,amsthm}
\usepackage{lastpage}
\usepackage{epigraph}
\usepackage{enumerate}
\usepackage{enumitem}
\usepackage{mathrsfs}
\usepackage{tikz}
\usepackage{pgfplots}
\pgfplotsset{compat=1.15}

\usetikzlibrary{arrows}
\usepackage{subfiles} % Insert the commands.tex file which contains the majority of the structure behind the template
\pgfplotsset{compat=1.15}

\newlist{questions}{enumerate}{3}
\setlist[questions]{label=\arabic*.}
\newcommand{\question}{\item}
\setlist[enumerate,1]{% (
leftmargin=*, itemsep=12pt, label={\textbf{\arabic*.)}}}

\newlist{partes}{enumerate}{3}
\setlist[partes]{label=(\alph*)}
\newcommand{\parte}{\item}
%---
\newlist{subpartes}{enumerate}{3}
\setlist[subpartes]{label=\roman*)}
\newcommand{\subparte}{\item}


\newcommand*\circled[1]{\tikz[baseline=(char.base)]{\node[shape=circle,draw,inner sep=2pt] (char) {#1};}}

\pagestyle{plain}
\newcommand{\instituto}{Universidad Sergio Arboleda}
\newcommand{\curso}{Matemáticas Discretas}
\newcommand{\professor}{Diego Villamizar}
\newcommand{\disciplina}{Matemáticas}
\newcommand{\titulo}{Taller 4}
\newcommand{\alumnoI}{Juan Sebastián Caballero Bernal}
\newcommand{\alumnoII}{Luz Ángela Orjuela Nieto}
\linespread{1.5}
\pagestyle{empty}
\newtheorem*{definition*}{Definición}
\newtheorem*{theorem*}{Teorema}
\newtheorem*{axiom*}{Postulado}
\newtheorem{theorem}{Teorema}[section]
\renewcommand*{\proofname}{\textbf{Demostración}}

\begin{document}
%%%%%%%%%%%%%%%%%%%%%%%%%%%%%%%%%%%%%%%%%%%%%%%%%%%%%%%%
%                      Emcabezado                     %
%%%%%%%%%%%%%%%%%%%%%%%%%%%%%%%%%%%%%%%%%%%%%%%%%%%%%%%%
\begin{table}[H]
\centering
\begin{tabular*}{\textwidth}{l@{\extracolsep{\fill}}l@{\extracolsep{\fill}}}
    \begin{tabular}[l]{@{}l@{}}
        \textbf{\instituto}\\
        \textbf{Disciplina: \disciplina}\\
        \textbf{Profesor: \professor}\\ 
    \end{tabular} & 
    \begin{tabular}[l]{@{}l@{}}
        {\curso}\\
        {\alumnoI}\\
    \end{tabular}
\end{tabular*}
\end{table}
\begin{center}
\rule[2ex]{\textwidth}{1pt}
{\Large{\titulo}}
\end{center}
\rule[2ex]{\textwidth}{1pt}
\begin{questions}[label=\protect\circled{\bfseries\arabic*}]
    \question Considere la siguiente sucesión: 
        $$F_n = \begin{cases} n &  n\leq 1\\
        F_{n-1}+F_{n-2} & n>1\end{cases}$$
    Calcule $F_8$, pruebe que $\Delta[F_x]=F_{x-1}$ y concluya, usando el T.F.C.D, que 
        $$\sum _{k = 0}^nF_k=F_{n+2}-1.$$

    \begin{proof}
        Para empezar, cálculando los primeros terminos de la sucesión:
        \begin{align*}
            F_0 &= 0\\
            F_1 &= 1\\
            F_2 &= 1\\
            F_3 &= 2\\
            F_4 &= 3\\
            F_5 &= 5\\
            F_6 &= 8\\
            F_7 &= 13\\
            F_8 &= 21
        \end{align*}

        Luego, evaluando la derivada de la sucesión para el caso en el cúal $x \ge 1$
        \begin{align*}
            \Delta[F_x] &= F_{x+1} - F_x\\
            &= F_x + F_{x-1} - F_x\\
            &= F_{x-1} 
        \end{align*}
        Notese que cuando $x = 0$ tendremos $\Delta[F_x] = 1$. Luego, gracias a ese resultado $\Delta[F_{x+1}] = F_x$ podremos evaluar la siguiente expresión:
        \begin{align*}
            \sum_{k = 0}^n F_k &= \sum_0^n F_x\\
            &= \sum_0^n \Delta[F_{x+1}]\\
            &= F_{x+1} |_{0}^{n+1}\\
            &= F_{n+2} - F_1\\
            &= F_{n+2} - 1
        \end{align*}
    \end{proof}

    \question Use el T.F.C.D para probar que si $c\geq 2$ entero, entonces
        $$\sum _{k = 0}^nc^k=\frac{c^{k+1}-1}{c-1}.$$

    \begin{proof}
        Note primero que:
        \begin{align*}
            \Delta\left[\frac{c^x}{c-1}\right] &= \frac{c^{x+1}}{c-1} - \frac{c^x}{c-1}\\
            &= \frac{c^{x+1}-c^x}{c-1}\\
            &= \frac{c^x(c-1)}{c-1}\\
            &= c^x
        \end{align*}
        Lo que permite evaluar la siguiente expresión:
        \begin{align*}
            \sum_{k = 0}^n c^k &= \sum_0^n c^x\\
            &= \sum_0^n \Delta\left[\frac{c^x}{c-1}\right]\\
            &= \frac{c^{n+1}}{c-1} - \frac{c^0}{c-1}\\
            &= \frac{c^{n+1}-1}{c-1}
        \end{align*}
    \end{proof}

    \question Pruebe que:
    \begin{align*}
        {n \brack 2} &= (n-1)! \cdot H_{n-1}
    \end{align*}
    \begin{proof}
        Recordemos que:
        \begin{align*}
            {n \brack 2} &= \sum_{x \in \binom{[n-1]}{n-2}} \prod_{x \in X} x
        \end{align*}

        Gracias a la formula que tenemos para cálcular binomiales, sabemos que:
        \begin{align*}
            \binom{n-1}{n-2} &= \frac{(n-1)!}{(n-1-(n-2))! (n-2)!}\\
            &= \frac{(n-1) (n-2)!}{(n-2)!}\\
            &= n-1
        \end{align*}
        Cada uno de los conjuntos que están en $\binom{[n-1]}{n-2}$ es $[n-1]$ sin un elemento. Luego, si queremos cálcular la productoria de los elementos de cada conjunto, sabemos que $\prod_{x \in [n-1]} x = (n-1)!$, por lo que cuando sacamos $i$ de $[n-1]$ la productoria será $\frac{(n-1)!}{i}$ para todo $i \in [n-1]$. Por lo que:
        \begin{align*}
            \sum_{x \in \binom{[n-1]}{n-2}} \prod_{x \in X} x &= \frac{(n-1)!}{1} + \frac{(n-1)!}{2} + \dots + \frac{(n-1)!}{(n-1)}\\
            &= (n-1)! \left[1 + \frac{1}{2} + \dots + \frac{1}{n-1}\right]\\
            &= (n-1)! \sum_{i = 1}^{n-1} \frac{1}{i}
        \end{align*}
        Luego, si hacemos un cambio de variable de forma que $k = i - 1$, tendremos que la sumatoria empezará en $0$, llegará hasta $n-2$ y se expresará como:
        \begin{align*}
            (n-1)! \sum_{i = 1}^{n-1} \frac{1}{i} &= (n-1)! \sum_{k = 0}^{n-2} \frac{1}{k+1}\\
            &= (n-1)! \sum_{k = 0}^{n-2} k^{\underline{-1}}\\
        \end{align*}
        Y si recordamos que
        \begin{align*}
            \sum _{k = 0}^nk^{\underline{-1}}=H_{n+1},
        \end{align*}
        entonces concluimos que:
        \begin{align*}
            {n \brack 2} &= (n-1)! \cdot H_{n-1}
        \end{align*}
    \end{proof}

    \question Calcule, usando el T.F.C.D, la expresión $$\sum _{k = 0}^n\frac{1}{(k+1)(k+2)}.$$
    \textbf{Hint:} ¿Qué es $\Delta\left [\frac{1}{x}\right ]$?
    \begin{proof}
        Note primero que:
        \begin{align*}
            \Delta\left[-\frac{1}{x+1}\right] &= -\frac{1}{x+2} + \frac{1}{x+1}\\
            &= \frac{-x-1 + x + 2}{(x+2)(x+1)}\\
            &= \frac{1}{(x+2)(x+1)}
        \end{align*}
        Lo que permite evaluar la expresión:
        \begin{align*}
            \sum _{k = 0}^n\frac{1}{(k+1)(k+2)} &= \sum_0^n \frac{1}{(x+1)(x+2)}\\
            &= \sum_0^n \Delta\left[-\frac{1}{x+1}\right]\\
            &= -\frac{1}{n+2} + \frac{1}{1}\\
            &= \frac{n+2}{n+2} - \frac{1}{n+2}\\
            &= \frac{n+1}{n+2} 
        \end{align*}
    \end{proof}
\end{questions}
\end{document}